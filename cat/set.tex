\documentclass{jsarticle}
\RequirePackage{amsmath,amssymb,amsthm, amscd, comment, multicol}
\usepackage[all]{xy}
\usepackage[dvipdfmx]{graphicx}
\usepackage{tikz-cd}
\input{../tex/theorems}
\input{../tex/symbols}

\title{集合と写像}
\author{梅崎 直也@unaoya}
\date{\today}
\begin{document}
\maketitle

まずは、集合や写像の言葉について復習する。
これは圏の基本的な概念を記述するために必要であり、また圏や関手に関しての基本的な例を与えるためにも必要である。

\begin{dfn}
このノートでの記号として、$0$以上の整数$n$にたいし集合$\{0,1,\ldots,n\}$のことを$[n]$と書くこととする。
\end{dfn}

二つの集合$X,Y$が与えられた時、
$X$の各要素$x\in X$に対して$Y$の要素$f(x)\in Y$を定めることにより写像$f\colon X\to Y$が定まる。
例えば$X=[1], Y=[1]$としたとき、その間の写像$f$は$0$の行き先$f(0)$を$0, 1$のいずれかとし、$1$の行き先$f(1)$を$0,1$のいずれかとしてやることで定義できる。
それらを全て書き出すと以下のようになる。
\[
\begin{cases}f_0(0)=0\\f_0(1)=0\end{cases}
\begin{cases}f_1(0)=0\\f_1(1)=1\end{cases}
\begin{cases}f_2(0)=1\\f_2(1)=0\end{cases}
\begin{cases}f_3(0)=1\\f_3(1)=1\end{cases}
\]

写像$X\to Y$全体の集合を$\Hom(X,Y)$と書く。
したがって上の記号を用いると
\[
\Hom([1],[1])=\{f_0,f_1,f_2,f_3\}
\]
である。

\begin{prob}
$0$以上の整数$n, m$にたいし$\Hom([n],[m])$がどのような集合になるか説明せよ。
\end{prob}
全て書き出すことはしないが、$\Hom([n],[m])$は$(m+1)^{n+1}$個の要素からなる集合であることを確認しよう。

任意の集合$X$について空集合$\emptyset$からの写像$\emptyset\to X$はただひとつ存在し、空でない集合$X$から空集合$\emptyset$への写像は存在しない。
つまり$\Hom(\emptyset,X)$はただ一つの要素からなる集合である。
また$X\neq\emptyset$であれば$\Hom(X,\emptyset)=\emptyset$である。

\subsection{自然な同一視}
$1$点集合$[0]=\{0\}$に対して$X$から$[0]$への写像全体$\Hom(X,[0])$がどのような集合になるかを考える。
$f:X \to [0]$は$X$の各要素を$[0]$の要素に対応させることで定まるが、$[0]$の要素は$0$のみなので、
全ての$x \in X$に対して$f(x)=0$とすることしかできず、$\Hom(X,[0])$はこの写像$f$ただ一つからなる集合$\{f\}$である。

\vspace{10pt}

次に$[0]$から$X$への写像全体$\Hom([0],X)$はどのような集合になるか。
$f:[0] \to X$は$0 \in [0]$の行き先を$X$のいずれかの要素に対応させることから定まる。
つまり$\Hom([0],X)$という集合の要素$f$は$f(0) = x \in X$として$x$を一つ決めることに対応し、
$\Hom([0],X)$という集合の要素と$X$という集合の要素はぴったり対応させることができる。

このようにして二つの集合$\Hom([0],X)$と$X$は同一視することができるが、
この同一視は単に要素がぴったり対応するという以上に、\textbf{自然な}同一視と呼ばれる特別な同一視である。

\begin{center}
自然な、とはどういう意味か?
\end{center}
というのをはっきりさせるのがこのノートでの一つの目標である。
その名にあるように、自然変換というのはこのような概念を数学的に定式化する一つの方法である。

\vspace{10pt}

ひとまず、ここでは次のように理解しよう。
例えば要素が二つからなる集合$\{a,b\}$と$[1]=\{0,1\}$は$a$と$0$、$b$と$1$を対応させることで同一視できるが、
$a$と$1$、$b$と$0$を対応させることでも同一視できる。
このどちらの同一視の方法がいいかを決める方法がなく、恣意的にどちらかを選ばなければならない。

一方で$X=\{a,b\}$と$\Hom([0],X)$は$a$に対して写像$0\mapsto a$を、$b$に対して写像$0\mapsto b$を対応させるという同一視の仕方が人の意思によらず、$X$自身によって自動的に決まっているような気がする。

さらに$Y=\{c,d,e\}, Z=\{f,g,h,i\}$のような集合に対しても同じやり方で$Y$と$\Hom([0],Y)$を同一視、$Z$と$\Hom([0],Z)$を同一視できる。
これらの同一視の方法は$X, Y, Z$といったそれぞれの集合について、全く同じやり方で、一斉に同一視のやり方を指定できている。

これはまだ説明していない圏論の言葉を使えば、次のように説明できる。

\begin{thm}
集合の圏から集合の圏への\textbf{関手}として、$X$に$X$を対応させる関手$\id$と、$X$に$\Hom([0],X)$を対応させる関手$\Hom([0],-)$の二つを考える。
この二つの関手$\id$と$\Hom([0],-)$の間には同型な\textbf{自然変換}が、$X$に対して$\Hom([0],X)$を対応させることで定まる。
\end{thm}

\subsection{写像の集合$\Hom$}
さて、上の例で現れた二つの集合$[0], X$の間の写像全体のなす集合$\Hom([0],X)$は圏論においてとても重要な対象である。
というわけで、圏論の本題に入る前にもう少しこのような集合に関して理解をしておこう。

\begin{dfn}[写像の合成]
二つの写像$f\colon X\to Y$と$g\colon Y\to Z$から新たな写像$g\circ f\colon X\to Z$を$x$にたいし$g(f(x))$を対応させることで定める。
これを$f$と$g$の合成とよぶ。
\end{dfn}

\begin{eg}
上で挙げた$\Hom([1],[1])$の例で言えば$f_0$と$f_2$の合成$f_0\circ f_2$は$0$を$f_0(f_2(0))=f_0(1)=0$に、$1$を$f_0(f_2(1))=f_0(0)=0$にうつすので$f_0\circ f_2=f_0$となる。同じように他の合成を計算してみると
\begin{itemize}
\item $f_1(f_2(0))=f_1(1)=1, f_1(f_2(1))=f_1(0)=0$となるので、$f_1\circ f_2=f_2$となる。
\item $f_2(f_2(0))=f_2(1)=0, f_2(f_2(1))=f_2(0)=1$となるので、$f_2\circ f_2=f_1$となる。
\item $f_3(f_2(0))=f_3(1)=1, f_3(f_2(1))=f_3(0)=1$となるので、$f_3\circ f_2=f_3$となる。
\end{itemize}
となる。
\end{eg}

\begin{prob}
$\Hom([1],[1])$における写像の合成がどのようになるか、全て計算せよ
\end{prob}

上の計算をしてみると、$\Hom([1], [1])$の各元$f_0, f_1, f_2, f_3$それぞれに$f_2$を合成するという操作により、
\begin{align*}
f_0\circ f_2=f_0\\
f_1 \circ f_2=f_2\\
f_2\circ f_2=f_1\\
f_3\circ f_2=f_3
\end{align*}
が新しく$\Hom([1], [1])$の元として定まっている。
これはつまり、集合$\Hom([1], [1])$から集合$\Hom([1], [1])$への写像を
\begin{align*}
f_0 \mapsto f_0\\
f_1 \mapsto f_2\\
f_2 \mapsto f_1\\
f_3 \mapsto f_3
\end{align*}
として定義していることになる。

一般的にかけば、写像$f:X \to Y$を用いて、写像$\Hom(Y,Z) \to \Hom(X,Z)$を$g \mapsto g \circ f$により定めている。
このようなものを$f$が\textbf{誘導する写像}などといい、しばしば$f^*$という記号を用いる。

\begin{rem}
ここでは写像の集合$\Hom(Y,Z)$から写像の集合$\Hom(X,Z)$への写像$f^*$を考えることになるので混乱してしまう。
$\Hom(Y,Z), \Hom(X,Z)$の要素は、ある時には写像であることを忘れて単なる集合の一要素と思い、ある時には実際に写像であったことを思い出す、というような視点の切り替えが必要になるので、慣れるまで何度か色々な計算をしてみてほしい。
\end{rem}

同じように写像$f:X \to Y$を用いて、写像$\Hom(Z,X) \to \Hom(Z,Y)$を$g \mapsto f \circ g$により定めることができる。
これも$f$が誘導する写像といい、こちらは$f_*$という記号を用いる。

一般的に定義の形で改めて述べておく。
\begin{dfn}
$f:X \to Y$にたいし、
写像$f_*:\Hom(Z,X)\to\Hom(Z,Y)$を$g:Z\to X$にたいし$f\circ g:Z\to Y$を対応させることで定める。
写像$f^*:\Hom(Y,Z)\to\Hom(X,Z)$を$g:Y\to Z$にたいし$g\circ f:X\to Z$を対応させることで定める。
\end{dfn}

\begin{eg}
$X=\{a,b,c\}$にたいし$\Hom(X,[1])$を考える。
$\Hom(X,[1])$の要素は$X$から$[1]$への写像$f:X \to [1]$であり、
これは$a, b, c$の三つに$0, 1$のいずれかを割り当てることで定まる。
実際には、$\Hom(X, [1])$は全部で$8$個の要素からなる集合である。
この集合を$f:X \to [1]$に対して$a, b, c$それぞれの行き先を並べることで$(f(a), f(b), f(c))$という$0,1$の三つ組と同一視しよう。

\vspace{5pt}

$f_2:[1] \to [1]$は$f_2(0)=1, f_2(1)=0$で定まる写像であった。
この$f_2$が誘導する写像$(f_2)_*:\Hom(X,[1]) \to \Hom(X, [1])$がどのような写像であるかみてみよう。

\vspace{5pt}

例えば$g:X \to [1]$を$g(a)=1, g(b)=0, g(c)=1$とする。
上の記法で言えば$g=(1,0,1)$ということである。
$(f_2)_*(g)$は$f_2$と$g$を合成することで得られる写像$g\circ f_2:X \to [1]$であり、$(f_2)_*(g)(a)=0, (f_2)_*(g)(b)=1, (f_2)_*(g)(c)=0$なる写像になる。
つまり、$(f_2)_*(g)=(0,1,0)$である。
他の写像についても同様に計算でき、$(f_2)_*$を三つ組で捉えれば$0$と$1$を入れ替える写像ということができる。

\vspace{5pt}

次に$Y=\{s,t\}$として、$f:Y \to X$を$f(s)=a, f(t)=c$と定める。
これに対し、$f^*:\Hom(X,[1]) \to \Hom(Y,[1])$を計算しよう。
上と同様に$\Hom(X,[1])$を$0,1$の三つ組とみなし、$\Hom(Y,[1])$を$0, 1$の二つ組とみなそう。
すると$f^*((0,1,0))=(0,0)$であり、$f^*((0,1,1))=(0,1)$である。
同様に他の$6$個の$\Hom(X,[1])$の要素の行き先も計算してみよう。
\end{eg}

\vspace{15pt}

このようにみると、$\Hom(-,X)$という操作は、集合$Y$に対して集合$\Hom(Y,X)$を定め、写像$f:Y \to Z$に対して写像$f^*:\Hom'Z,X) \to \Hom(Y,X)$を定めている。
$\Hom(X,-)$も同様である。
これらは後で見るように、集合の圏から集合の圏への関手の例になっている。

\subsection{冪集合}
関手のもう一つ重要な例として、集合の冪集合というものを考えよう。
集合$X$から、その部分集合を全て集めて新しい集合$P(X)$を作る。
これを$X$の冪集合という。

\begin{dfn}[冪集合]
集合$X$に対して、その冪集合$P(X)$とは、$X$の部分集合全体からなる集合のことである。
\end{dfn}

\begin{eg}
$X=\{0,1\}$の冪集合は
\begin{align*}
P(X)=\{\emptyset, \{0\},\{1\},X\}
\end{align*}
となる。

$X=\{a,b,c\}$であれば、$X$の冪集合は
\begin{align*}
P(X)=\{\emptyset, \{a\}, \{b\}, \{c\}, \{a,b\}, \{a,c\}, \{b,c\}, X\}
\end{align*}
と$8$個の要素からなる集合である。
\end{eg}

\begin{prob}
$0$以上の整数$n$にたいし$P([n])$はどのような集合か。
\end{prob}


写像$f\colon X\to Y$にたいし写像$P(f)\colon P(Y)\to P(X)$を$V\subset Y$に対し$P(f)(V)=\{x\in X\mid f(x)\in V\}\subset X$とすることで定める。
通常この$P(f)(V)$は$f^{-1}(V)$と書かれるもので、$V$の$f$による逆像と呼ばれる。
ここでは$P$が関手であるという見方を強調するため$P(f)$という書き方を使うことにする。

\begin{prob}
$X=\{a,b,c\}, Y=\{s,t\}$のとき、$f:Y \to X$に対して$P(f)$がどのような写像$P(X) \to P(Y)$になるか計算しよう。
例えば上の例でやったのと同様に$f:X \to Y$を$f(s)=a, f(t)=c$とさだめよう。
すると、$P(X)$のそれぞれについて、$P(f)$がどのようになるかを見てみよう。

例えば$P(f)(\{a,b\})$であれば$Y$の要素それぞれについて$f$でうつった先が$\{a,b\}$に入っているかを確かめることになる。
$f(s)=a \in \{a,b\}, f(t)=c\notin\{a,b\}$なので、$P(f)(\{a,b\})=\{s\}$である。
\end{prob}

このような例を計算していると、$P(X)$と$\Hom(X,[1])$に何か関係があるということに気づくかもしれない。
実際、部分集合$A \subset X$から写像$X \to [1]$を$A$に入っていれば$0$を、入っていなければ$1$を対応させるということで定めることができ、これにより$P(X)$と$\Hom(X,[1])$は一対一に対応する。
この対応も自然な対応である。

つまり$P(X)$と$\Hom(X,[1])$を対応させるのと同じ方法で$P(Y)$と$\Hom(Y,[1])$を対応させることができ、さらに$f:X \to Y$から$P(Y) \to P(X), \Hom(X,[1]) \to \Hom(Y,[1])$が一斉に決まる。
これは後で見るように、関手$P$と関手$\Hom(-,[1])$の間の自然変換を与えていることになる。

\subsection{集合の直積}

\begin{dfn}[直積]
二つの集合$X, Y$から新しい集合$X \times Y$を、$X$の要素と$Y$の要素を対にしたもの全体の集合として定める。
これを$X$と$Y$の直積という。

同様にして三つの集合の直積$X \times Y \times Z$などを定義することもできる。
\end{dfn}

\begin{eg}
$[1] \times [1]=\{(0,0), (1,0), (0,1), (1,1)\}$である。
\end{eg}

\begin{eg}
$xy$座標平面の点全体$\R^2=\{(x,y)\vert x\in\R, y\in\R\}$である。
\end{eg}

直積には\emph{自然に}二つの写像$pr_1:X \times Y \to X, pr_2:X \times Y \to Y$が定まる。
これらはそれぞれ対の左側の要素、右側の要素を取り出す写像である。
\end{document}