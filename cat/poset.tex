\documentclass{jsarticle}
\RequirePackage{amsmath,amssymb,amsthm, amscd, comment, multicol}
\usepackage[all]{xy}
\usepackage[dvipdfmx]{graphicx}
\usepackage{tikz-cd}
\input{../../tex/theorems}
\input{../../tex/symbols}

\title{順序集合}
\author{@unaoya}
\date{\today}
\begin{document}
\maketitle

ここでは、圏についての簡単な例を与えるために順序集合について紹介する。
順序集合というのは、単に物の集まりとしての集合を考えるだけでなく、集合の要素の間に順序関係を適切に定めたものである。

例えば、整数全体のなす集合$\{\ldots,-2,-1,0, 1, 2, \ldots\}$は通常の大小関係によって順序が定まっており、
それにより順序集合とみなすことができる。

順序集合はそれぞれ圏とみなすことができる。
例えば上で述べたような整数全体のなす順序集合から、それに対応する圏を考えることができるし、
有限個の整数$[n]=\{0,1,2,\ldots,n\}$を通常の大小関係によって順序集合とみなしたものも、それに対応する圏を考えることができる。

「集合を全て集めたもの」のようなとても巨大な圏もあれば、「有限個の点を矢印で結んだもの」のような絵にかける圏もある。
一つの順序集合は、後者のような小さな圏の一つの例であるとみなすことができる。

\section{順序集合の定義と例}

まずは順序集合の定義を確認する。
大小関係が満たすべき性質を抽象化し、以下のように順序集合を定義する。
\begin{dfn}[順序集合]
順序集合$(X,\leq)$とは集合$X$とその要素$x, y, z$たちの関係$\leq$であって、
\begin{enumerate}
\item $x\leq x$
\item $x\leq y$かつ$y\leq x$なら$x=y$
\item $x\leq y$かつ$y\leq z$なら$x\leq z$
\end{enumerate}
を満たすもの。
\end{dfn}

\begin{eg}
例えば上で述べた有限個の整数とその大小による順序集合といった場合、
$[2]$であれば$X=\{0,1,2\}$であり、関係としては$0\leq0,0\leq1,0\leq2,1\leq1,1\leq2,2\leq2$が全て。

一般の$[n]$では、関係として$0\leq 0, 0\leq1,\ldots,0\leq n,1\leq1,1\leq,2,\ldots,1\leq n,\ldots,n\leq n$を定めていることになる。
\end{eg}

つまり、大小関係が成り立つような組み合わせを全て列挙し、それをまとめて$\leq$という記号で表したと考えればよい。

\begin{rem}
上の2つ目の条件は、
\begin{center}
もし$x\neq y$であれば$x\leq y, y\leq x$のいずれかのみが成り立つか、$x$と$y$の間には順序が定義されない
\end{center}
ということである。
一般には順序集合には全ての元の間に順序関係が定まっている必要はないということに注意する。
\end{rem}

他にも次のような例を考える。
\begin{eg}
$1$以上の整数の集合$\mathbb{N}_+$上に$n$が$m$を割り切る時$n\leq m$として順序を定めたもの。
例えば$2\leq 4, 3\leq 12$などとなるが$2$と$3$の間には順序関係はない。
\end{eg}

\begin{eg}
集合$X$にたいし$P(X)$を冪集合とし、包含関係によって$P(X)$に順序を定める。
つまり$U\subset V$のとき$U\leq V$と定める。

例えば$X=[1]$であれば$P(X)=\{\emptyset, \{0\}, \{1\}, \{0,1\}\}$であり、
ここで定まる順序関係は$\emptyset \leq \emptyset, \emptyset \leq \{0\}, \emptyset \leq \{1\}, \emptyset \leq \{0,1\}, \{0\} \leq \{0\}, \{0\} \leq \{0, 1\}, \{1\} \leq \{1\}, \{1\} \leq \{0,1\}, \{0, 1\} \leq \{0, 1\}$が全て。
これ以外には順序は定まらない。
例えば$\{0\}$と$\{1\}$の間には関係はない。
\end{eg}

次の例はつまらないものだが、このような例を考えることで色々なことが一般的に扱いやすい。
\begin{eg}
集合$X$と自明な順序$x\leq x$のみの順序集合。
つまり異なる二つの元の間には順序関係はない。

例えば$X=[1]$であれば、これに対して$0\leq0, 1\leq1$のみが順序関係であるとして、順序集合だとする。
\end{eg}

改めて強調するが、必ずしも全ての元の間に順序関係が成り立つ必要はない。

\section{順序集合の射}
次に二つの順序集合を結び付けるために、それらの間の写像を考える。
ここで射という言葉を使ったのは、単なる集合の写像ではなく、順序集合が持つ順序という関係がちゃんと対応するように定まった写像ということである。

圏論においてはこのような特別な写像だけを考えることが多く、それを射と呼ぶ。
つまり、ここで定義するのは順序集合たちのなす圏における射という言い方ができる。

\begin{dfn}
順序集合$X, Y$の間の射$f:X\to Y$とは、$x, x'\in X$で$x\leq x'$であれば$Y$において$f(x)\leq f(x')$となる写像のこと。
つまり、$X$において順序関係を持つ二つの要素を$f$で写すと、$Y$においても順序関係を持っているように定まっている写像である。
\end{dfn}

\begin{eg}
集合$[1]$を通常の大小によって順序集合とみなす。
前に列挙した写像$[1]\to [1]$について、$f_2$以外は順序集合の射であり、
$f_2$は$0\leq 1, f_2(0)=1\geq f_2(1)=0$のため順序集合の射にならない。

一方$[1]$に自明な順序により順序集合とみなす。
すると、全ての写像は順序集合の射二なる。
\end{eg}

この例で見たように、集合の写像$f:X \to Y$について、$X, Y$にどのような順序を定めるかによって順序集合の射になったりならなかったりする。

\begin{eg}
冪集合を上の例のように包含関係により順序集合とみなす。
集合の写像$f:X \to Y$から定まる$P(f):P(Y) \to P(X)$は順序集合の射である。
つまり、$Y$の部分集合$U, V \subset Y$が$U\subset V$をみたすならば$P(f)(U) \subset P(f)(V)$である。
\end{eg}

\section{順序集合における米田の補題}
次に紹介するのは、順序集合における米田の補題と言うべきものである。
順序集合における要素は、その順序集合全ての要素との関係により決定されるというもの。

例えばはじめに紹介した$[n]$のような全ての要素が一列に並んでいる場合、ある要素$x$はそれ以下の要素全てを列挙することで$x$を特定することができる。
$[3]$であれば$0$に対して$\{0\}$を対応させ、$1$に対して$\{0,1\}$、$2$に対して$\{0,1,2\}$、$3$に対して$\{0,1,2,3\}$といったようにすればよい。

$P([1])$のような場合でも、$\emptyset$に対して$\{\emptyset\}$、$\{0\}$に対して$\{\emptyset, \{0\}\}$、$\{1\}$に対して$\{\emptyset, \{1\}\}$、$[1]$に対して$P([1])$となるから、この対応により$P([1])$の要素が全て区別できる。

これを一般化するため以下のように考えよう。
$x\in X$に対し関数$h_x:X\to [1]$を$h_x(y)=\Hom_X(y,x)\mbox{の元の個数}$として定める。
つまり
\[
h_x(y)=\begin{cases}1&y\leq x\\0&\mbox{それ以外}\end{cases}
\]
この関数で$1$になるものを全て集めるということが、上で述べたそれ以下の要素を全て列挙するということと同じことになる。

従って、以下のようなことが証明できる。
\begin{prob}
$x\leq y$であることと任意の$z\in X$で$h_x(z)\leq h_y(z)$であることが同値であることを示せ。
特に$x=y$であることと任意の$z\in X$で$h_x(z)=h_y(z)$であることが同値である。

ここで$h_x(z)\leq h_z(y)$は$0\leq0, 0\leq1, 1\leq1$なる順序(つまり通常の大小関係)で定まっている。
\end{prob}

$x\leq y$としよう。
$h_x(z)=0$なら常に$h_x(z)\leq h_y(z)$である。
$h_x(z)=1$なら$z\leq x$であり、$z\leq y$となるので$h_y(z)=1$である。
従って、この場合任意の$z \in X$について$h_x(z)\leq h_y(z)$であることがわかった。

同様にして残りの部分も証明できる。

\section{順序集合の図}

順序集合$X$から以下の手順で図を描くことができる。
\begin{enumerate}
\item まず$X$の元に対応する頂点を描く。
\item 次に$x\leq y$のとき$x$から$y$に向かう矢印を描く。
\end{enumerate}

例えば$X=[1]$で自明な順序のみの場合、
\begin{tikzcd}
0 \ar[loop] &  1 \ar[loop]
\end{tikzcd}

$X=[1]$に通常の整数の順序を入れた場合、
\begin{tikzcd}
0 \ar[loop] \ar[r] &  1 \ar[loop]
\end{tikzcd}

$X=P([1])$の場合、
\begin{tikzcd}
& \{0,1\} \ar[loop] & \\
\{0\} \ar[loop] \ar[ur] & &  \{1\} \ar[ul] \ar[loop]\\
& \emptyset \ar[ur] \ar[uu] \ar[ul] \ar[loop]&
\end{tikzcd}

$X=\mathbb{N}_+$の場合、一部を書くと
\begin{tikzcd}
1 \ar[loop] \ar[d] \ar[dr] \ar[drr] \ar[drrr] \ar[dd] \ar[ddr] \ar[ddrr] \ar[ddrrr] \ar[ddd]\\
2 \ar[loop] \ar[d] \ar[dr] \ar[drrr] \ar[dd] &3 \ar[loop] \ar[d] \ar[dr] &5 \ar[loop] \ar[dr] &7 \ar[loop]\\
4 \ar[d] \ar[loop] &6 \ar[loop] &9 \ar[loop] &10 \ar[loop] \\
8 \ar[loop] \\
\end{tikzcd}

のようになる。
これらの絵は「幾つかの点とそれを結ぶ矢印たちの集まりで適当な条件をみたすもの」であり、これは圏の例を与えている。
\begin{itemize}
\item 点たちを圏の対象という。
ここでは対象を集めた集合は$X$である。
\item また二点$x, y$の間の矢印の集合を$\Hom_X(x,y)$とかく。
ここでは
\[
\Hom_X(x,y)=\begin{cases}\{\to\} &x\leq y\\ \emptyset &\mbox{そうでない時}\end{cases}
\]
\end{itemize}

順序集合の定義によれば、
\begin{enumerate}
\item それぞれの点について、始点と終点が自分自身であるような矢印がただ一つある
\item 異なる二点の間の矢印があるとすればどちらか向きに一つだけある
\item 二つの矢印をつないだ矢印がある
\end{enumerate}
という絵を描くことと順序集合を考えることは全く同等のことである。

順序集合の間の射をこの絵を使って考えると、点を点にうつし矢印を矢印にうつすものになる。
これが関手の例を与えている。

\end{document}