\documentclass[uplatex]{jsarticle}
\RequirePackage{amsmath,amssymb,amsthm, amscd, comment, multicol}
\usepackage[all]{xy}
\usepackage[dvipdfmx]{graphicx}
\usepackage{tikz-cd}
\input{../tex/theorems}
\input{../tex/symbols}

\title{圏入門その1 - 自然変換と米田の補題}
\author{梅崎 直也@unaoya}
\date{\today}
\begin{document}
\maketitle

\section{はじめに}

圏についての基本的な話題を、できるだけ予備知識を仮定せずに紹介するのがこのノートの目的です。
全部でその3まで、内容は
\begin{enumerate}
\item 自然変換と米田の補題
\item 普遍性と極限
\item 随伴
\end{enumerate}
の予定です。

その1では、米田の補題を証明することを目標に、圏、関手、自然変換の定義と基本的な例について書きました。
さらに米田の補題の応用としてde Morganの定理を証明します。

\section{集合と写像}
まずは、集合や写像の言葉について復習する。
これは圏の基本的な概念を記述するために必要であり、また圏や関手に関しての基本的な例を与えるためにも必要である。

\begin{dfn}
このノートでの記号として、$0$以上の整数$n$にたいし集合$\{0,1,\ldots,n\}$のことを$[n]$と書くこととする。
\end{dfn}

二つの集合$X,Y$が与えられた時、
$X$の各要素$x\in X$に対して$Y$の要素$f(x)\in Y$を定めることにより写像$f\colon X\to Y$が定まる。
例えば$X=[1], Y=[1]$としたとき、その間の写像$f$は$0$の行き先$f(0)$を$0, 1$のいずれかとし、$1$の行き先$f(1)$を$0,1$のいずれかとしてやることで定義できる。
それらを全て書き出すと以下のようになる。
\[
\begin{cases}f_0(0)=0\\f_0(1)=0\end{cases}
\begin{cases}f_1(0)=0\\f_1(1)=1\end{cases}
\begin{cases}f_2(0)=1\\f_2(1)=0\end{cases}
\begin{cases}f_3(0)=1\\f_3(1)=1\end{cases}
\]

写像$X\to Y$全体の集合を$\Hom(X,Y)$と書く。
したがって上の記号を用いると
\[
\Hom([1],[1])=\{f_0,f_1,f_2,f_3\}
\]
である。

\begin{prob}
$0$以上の整数$n, m$にたいし$\Hom([n],[m])$がどのような集合になるか説明せよ。
\end{prob}
全て書き出すことはしないが、$\Hom([n],[m])$は$(m+1)^{n+1}$個の要素からなる集合であることを確認しよう。

任意の集合$X$について空集合$\emptyset$からの写像$\emptyset\to X$はただひとつ存在し、空でない集合$X$から空集合$\emptyset$への写像は存在しない。
つまり$\Hom(\emptyset,X)$はただ一つの要素からなる集合である。
また$X\neq\emptyset$であれば$\Hom(X,\emptyset)=\emptyset$である。

\subsection{自然な同一視}
$1$点集合$[0]=\{0\}$に対して$X$から$[0]$への写像全体$\Hom(X,[0])$がどのような集合になるかを考える。
$f:X \to [0]$は$X$の各要素を$[0]$の要素に対応させることで定まるが、$[0]$の要素は$0$のみなので、
全ての$x \in X$に対して$f(x)=0$とすることしかできず、$\Hom(X,[0])$はこの写像$f$ただ一つからなる集合$\{f\}$である。

\vspace{10pt}

次に$[0]$から$X$への写像全体$\Hom([0],X)$はどのような集合になるか。
$f:[0] \to X$は$0 \in [0]$の行き先を$X$のいずれかの要素に対応させることから定まる。
つまり$\Hom([0],X)$という集合の要素$f$は$f(0) = x \in X$として$x$を一つ決めることに対応し、
$\Hom([0],X)$という集合の要素と$X$という集合の要素はぴったり対応させることができる。

このようにして二つの集合$\Hom([0],X)$と$X$は同一視することができるが、
この同一視は単に要素がぴったり対応するという以上に、\textbf{自然な}同一視と呼ばれる特別な同一視である。

\begin{center}
自然な、とはどういう意味か?
\end{center}
というのをはっきりさせるのがこのノートでの一つの目標である。
その名にあるように、自然変換というのはこのような概念を数学的に定式化する一つの方法である。

\vspace{10pt}

ひとまず、ここでは次のように理解しよう。
例えば要素が二つからなる集合$\{a,b\}$と$[1]=\{0,1\}$は$a$と$0$、$b$と$1$を対応させることで同一視できるが、
$a$と$1$、$b$と$0$を対応させることでも同一視できる。
このどちらの同一視の方法がいいかを決める方法がなく、恣意的にどちらかを選ばなければならない。

一方で$X=\{a,b\}$と$\Hom([0],X)$は$a$に対して写像$0\mapsto a$を、$b$に対して写像$0\mapsto b$を対応させるという同一視の仕方が人の意思によらず、$X$自身によって自動的に決まっているような気がする。

さらに$Y=\{c,d,e\}, Z=\{f,g,h,i\}$のような集合に対しても同じやり方で$Y$と$\Hom([0],Y)$を同一視、$Z$と$\Hom([0],Z)$を同一視できる。
これらの同一視の方法は$X, Y, Z$といったそれぞれの集合について、全く同じやり方で、一斉に同一視のやり方を指定できている。

これはまだ説明していない圏論の言葉を使えば、次のように説明できる。

\begin{thm}
集合の圏から集合の圏への\textbf{関手}として、$X$に$X$を対応させる関手$\id$と、$X$に$\Hom([0],X)$を対応させる関手$\Hom([0],-)$の二つを考える。
この二つの関手$\id$と$\Hom([0],-)$の間には同型な\textbf{自然変換}が、$X$に対して$\Hom([0],X)$を対応させることで定まる。
\end{thm}

\subsection{写像の集合$\Hom$}
さて、上の例で現れた二つの集合$[0], X$の間の写像全体のなす集合$\Hom([0],X)$は圏論においてとても重要な対象である。
というわけで、圏論の本題に入る前にもう少しこのような集合に関して理解をしておこう。

\begin{dfn}[写像の合成]
二つの写像$f\colon X\to Y$と$g\colon Y\to Z$から新たな写像$g\circ f\colon X\to Z$を$x$にたいし$g(f(x))$を対応させることで定める。
これを$f$と$g$の合成とよぶ。
\end{dfn}

\begin{eg}
上で挙げた$\Hom([1],[1])$の例で言えば$f_0$と$f_2$の合成$f_0\circ f_2$は$0$を$f_0(f_2(0))=f_0(1)=0$に、$1$を$f_0(f_2(1))=f_0(0)=0$にうつすので$f_0\circ f_2=f_0$となる。同じように他の合成を計算してみると
\begin{itemize}
\item $f_1(f_2(0))=f_1(1)=1, f_1(f_2(1))=f_1(0)=0$となるので、$f_1\circ f_2=f_2$となる。
\item $f_2(f_2(0))=f_2(1)=0, f_2(f_2(1))=f_2(0)=1$となるので、$f_2\circ f_2=f_1$となる。
\item $f_3(f_2(0))=f_3(1)=1, f_3(f_2(1))=f_3(0)=1$となるので、$f_3\circ f_2=f_3$となる。
\end{itemize}
となる。
\end{eg}

\begin{prob}
$\Hom([1],[1])$における写像の合成がどのようになるか、全て計算せよ
\end{prob}

上の計算をしてみると、$\Hom([1], [1])$の各元$f_0, f_1, f_2, f_3$それぞれに$f_2$を合成するという操作により、
\begin{align*}
f_0\circ f_2=f_0\\
f_1 \circ f_2=f_2\\
f_2\circ f_2=f_1\\
f_3\circ f_2=f_3
\end{align*}
が新しく$\Hom([1], [1])$の元として定まっている。
これはつまり、集合$\Hom([1], [1])$から集合$\Hom([1], [1])$への写像を
\begin{align*}
f_0 \mapsto f_0\\
f_1 \mapsto f_2\\
f_2 \mapsto f_1\\
f_3 \mapsto f_3
\end{align*}
として定義していることになる。

一般的にかけば、写像$f:X \to Y$を用いて、写像$\Hom(Y,Z) \to \Hom(X,Z)$を$g \mapsto g \circ f$により定めている。
このようなものを$f$が\textbf{誘導する写像}などといい、しばしば$f^*$という記号を用いる。

\begin{rem}
ここでは写像の集合$\Hom(Y,Z)$から写像の集合$\Hom(X,Z)$への写像$f^*$を考えることになるので混乱してしまう。
$\Hom(Y,Z), \Hom(X,Z)$の要素は、ある時には写像であることを忘れて単なる集合の一要素と思い、ある時には実際に写像であったことを思い出す、というような視点の切り替えが必要になるので、慣れるまで何度か色々な計算をしてみてほしい。
\end{rem}

同じように写像$f:X \to Y$を用いて、写像$\Hom(Z,X) \to \Hom(Z,Y)$を$g \mapsto f \circ g$により定めることができる。
これも$f$が誘導する写像といい、こちらは$f_*$という記号を用いる。

一般的に定義の形で改めて述べておく。
\begin{dfn}
$f:X \to Y$にたいし、
写像$f_*:\Hom(Z,X)\to\Hom(Z,Y)$を$g:Z\to X$にたいし$f\circ g:Z\to Y$を対応させることで定める。
写像$f^*:\Hom(Y,Z)\to\Hom(X,Z)$を$g:Y\to Z$にたいし$g\circ f:X\to Z$を対応させることで定める。
\end{dfn}

\begin{eg}
$X=\{a,b,c\}$にたいし$\Hom(X,[1])$を考える。
$\Hom(X,[1])$の要素は$X$から$[1]$への写像$f:X \to [1]$であり、
これは$a, b, c$の三つに$0, 1$のいずれかを割り当てることで定まる。
実際には、$\Hom(X, [1])$は全部で$8$個の要素からなる集合である。
この集合を$f:X \to [1]$に対して$a, b, c$それぞれの行き先を並べることで$(f(a), f(b), f(c))$という$0,1$の三つ組と同一視しよう。

\vspace{5pt}

$f_2:[1] \to [1]$は$f_2(0)=1, f_2(1)=0$で定まる写像であった。
この$f_2$が誘導する写像$(f_2)_*:\Hom(X,[1]) \to \Hom(X, [1])$がどのような写像であるかみてみよう。

\vspace{5pt}

例えば$g:X \to [1]$を$g(a)=1, g(b)=0, g(c)=1$とする。
上の記法で言えば$g=(1,0,1)$ということである。
$(f_2)_*(g)$は$f_2$と$g$を合成することで得られる写像$g\circ f_2:X \to [1]$であり、$(f_2)_*(g)(a)=0, (f_2)_*(g)(b)=1, (f_2)_*(g)(c)=0$なる写像になる。
つまり、$(f_2)_*(g)=(0,1,0)$である。
他の写像についても同様に計算でき、$(f_2)_*$を三つ組で捉えれば$0$と$1$を入れ替える写像ということができる。

\vspace{5pt}

次に$Y=\{s,t\}$として、$f:Y \to X$を$f(s)=a, f(t)=c$と定める。
これに対し、$f^*:\Hom(X,[1]) \to \Hom(Y,[1])$を計算しよう。
上と同様に$\Hom(X,[1])$を$0,1$の三つ組とみなし、$\Hom(Y,[1])$を$0, 1$の二つ組とみなそう。
すると$f^*((0,1,0))=(0,0)$であり、$f^*((0,1,1))=(0,1)$である。
同様に他の$6$個の$\Hom(X,[1])$の要素の行き先も計算してみよう。
\end{eg}

\vspace{15pt}

このようにみると、$\Hom(-,X)$という操作は、集合$Y$に対して集合$\Hom(Y,X)$を定め、写像$f:Y \to Z$に対して写像$f^*:\Hom'Z,X) \to \Hom(Y,X)$を定めている。
$\Hom(X,-)$も同様である。
これらは後で見るように、集合の圏から集合の圏への関手の例になっている。

\subsection{冪集合}
関手のもう一つ重要な例として、集合の冪集合というものを考えよう。
集合$X$から、その部分集合を全て集めて新しい集合$P(X)$を作る。
これを$X$の冪集合という。

\begin{dfn}[冪集合]
集合$X$に対して、その冪集合$P(X)$とは、$X$の部分集合全体からなる集合のことである。
\end{dfn}

\begin{eg}
$X=\{0,1\}$の冪集合は
\begin{align*}
P(X)=\{\emptyset, \{0\},\{1\},X\}
\end{align*}
となる。

$X=\{a,b,c\}$であれば、$X$の冪集合は
\begin{align*}
P(X)=\{\emptyset, \{a\}, \{b\}, \{c\}, \{a,b\}, \{a,c\}, \{b,c\}, X\}
\end{align*}
と$8$個の要素からなる集合である。
\end{eg}

\begin{prob}
$0$以上の整数$n$にたいし$P([n])$はどのような集合か。
\end{prob}


写像$f\colon X\to Y$にたいし写像$P(f)\colon P(Y)\to P(X)$を$V\subset Y$に対し$P(f)(V)=\{x\in X\mid f(x)\in V\}\subset X$とすることで定める。
通常この$P(f)(V)$は$f^{-1}(V)$と書かれるもので、$V$の$f$による逆像と呼ばれる。
ここでは$P$が関手であるという見方を強調するため$P(f)$という書き方を使うことにする。

\begin{prob}
$X=\{a,b,c\}, Y=\{s,t\}$のとき、$f:Y \to X$に対して$P(f)$がどのような写像$P(X) \to P(Y)$になるか計算しよう。
例えば上の例でやったのと同様に$f:X \to Y$を$f(s)=a, f(t)=c$とさだめよう。
すると、$P(X)$のそれぞれについて、$P(f)$がどのようになるかを見てみよう。

例えば$P(f)(\{a,b\})$であれば$Y$の要素それぞれについて$f$でうつった先が$\{a,b\}$に入っているかを確かめることになる。
$f(s)=a \in \{a,b\}, f(t)=c\notin\{a,b\}$なので、$P(f)(\{a,b\})=\{s\}$である。
\end{prob}

このような例を計算していると、$P(X)$と$\Hom(X,[1])$に何か関係があるということに気づくかもしれない。
実際、部分集合$A \subset X$から写像$X \to [1]$を$A$に入っていれば$0$を、入っていなければ$1$を対応させるということで定めることができ、これにより$P(X)$と$\Hom(X,[1])$は一対一に対応する。
この対応も自然な対応である。

つまり$P(X)$と$\Hom(X,[1])$を対応させるのと同じ方法で$P(Y)$と$\Hom(Y,[1])$を対応させることができ、さらに$f:X \to Y$から$P(Y) \to P(X), \Hom(X,[1]) \to \Hom(Y,[1])$が一斉に決まる。
これは後で見るように、関手$P$と関手$\Hom(-,[1])$の間の自然変換を与えていることになる。

\subsection{集合の直積}

\begin{dfn}[直積]
二つの集合$X, Y$から新しい集合$X \times Y$を、$X$の要素と$Y$の要素を対にしたもの全体の集合として定める。
これを$X$と$Y$の直積という。

同様にして三つの集合の直積$X \times Y \times Z$などを定義することもできる。
\end{dfn}

\begin{eg}
$[1] \times [1]=\{(0,0), (1,0), (0,1), (1,1)\}$である。
\end{eg}

\begin{eg}
$xy$座標平面の点全体$\R^2=\{(x,y)\vert x\in\R, y\in\R\}$である。
\end{eg}

直積には\emph{自然に}二つの写像$pr_1:X \times Y \to X, pr_2:X \times Y \to Y$が定まる。
これらはそれぞれ対の左側の要素、右側の要素を取り出す写像である。


\section{順序集合}
ここでは、圏についての簡単な例を与えるために順序集合について紹介する。
順序集合というのは、単に物の集まりとしての集合を考えるだけでなく、集合の要素の間に順序関係を適切に定めたものである。

例えば、整数全体のなす集合$\{\ldots,-2,-1,0, 1, 2, \ldots\}$は通常の大小関係によって順序が定まっており、
それにより順序集合とみなすことができる。

順序集合はそれぞれ圏とみなすことができる。
例えば上で述べたような整数全体のなす順序集合から、それに対応する圏を考えることができるし、
有限個の整数$[n]=\{0,1,2,\ldots,n\}$を通常の大小関係によって順序集合とみなしたものも、それに対応する圏を考えることができる。

「集合を全て集めたもの」のようなとても巨大な圏もあれば、「有限個の点を矢印で結んだもの」のような絵にかける圏もある。
一つの順序集合は、後者のような小さな圏の一つの例であるとみなすことができる。

\subsection{順序集合の定義と例}

まずは順序集合の定義を確認する。
大小関係が満たすべき性質を抽象化し、以下のように順序集合を定義する。
\begin{dfn}[順序集合]
順序集合$(X,\leq)$とは集合$X$とその要素$x, y, z$たちの関係$\leq$であって、
\begin{enumerate}
\item $x\leq x$
\item $x\leq y$かつ$y\leq x$なら$x=y$
\item $x\leq y$かつ$y\leq z$なら$x\leq z$
\end{enumerate}
を満たすもの。
\end{dfn}

\begin{eg}
例えば上で述べた有限個の整数とその大小による順序集合といった場合、
$[2]$であれば$X=\{0,1,2\}$であり、関係としては$0\leq0,0\leq1,0\leq2,1\leq1,1\leq2,2\leq2$が全て。

一般の$[n]$では、関係として$0\leq 0, 0\leq1,\ldots,0\leq n,1\leq1,1\leq,2,\ldots,1\leq n,\ldots,n\leq n$を定めていることになる。
\end{eg}

つまり、大小関係が成り立つような組み合わせを全て列挙し、それをまとめて$\leq$という記号で表したと考えればよい。

\begin{rem}
上の2つ目の条件は、
\begin{center}
もし$x\neq y$であれば$x\leq y, y\leq x$のいずれかのみが成り立つか、$x$と$y$の間には順序が定義されない
\end{center}
ということである。
一般には順序集合には全ての元の間に順序関係が定まっている必要はないということに注意する。
\end{rem}

他にも次のような例を考える。
\begin{eg}
$1$以上の整数の集合$\mathbb{N}_+$上に$n$が$m$を割り切る時$n\leq m$として順序を定めたもの。
例えば$2\leq 4, 3\leq 12$などとなるが$2$と$3$の間には順序関係はない。
\end{eg}

\begin{eg}
集合$X$にたいし$P(X)$を冪集合とし、包含関係によって$P(X)$に順序を定める。
つまり$U\subset V$のとき$U\leq V$と定める。

例えば$X=[1]$であれば$P(X)=\{\emptyset, \{0\}, \{1\}, \{0,1\}\}$であり、
ここで定まる順序関係は$\emptyset \leq \emptyset, \emptyset \leq \{0\}, \emptyset \leq \{1\}, \emptyset \leq \{0,1\}, \{0\} \leq \{0\}, \{0\} \leq \{0, 1\}, \{1\} \leq \{1\}, \{1\} \leq \{0,1\}, \{0, 1\} \leq \{0, 1\}$が全て。
これ以外には順序は定まらない。
例えば$\{0\}$と$\{1\}$の間には関係はない。
\end{eg}

次の例はつまらないものだが、このような例を考えることで色々なことが一般的に扱いやすい。
\begin{eg}
集合$X$と自明な順序$x\leq x$のみの順序集合。
つまり異なる二つの元の間には順序関係はない。

例えば$X=[1]$であれば、これに対して$0\leq0, 1\leq1$のみが順序関係であるとして、順序集合だとする。
\end{eg}

改めて強調するが、必ずしも全ての元の間に順序関係が成り立つ必要はない。

\subsection{順序集合の射}
次に二つの順序集合を結び付けるために、それらの間の写像を考える。
ここで射という言葉を使ったのは、単なる集合の写像ではなく、順序集合が持つ順序という関係がちゃんと対応するように定まった写像ということである。

圏論においてはこのような特別な写像だけを考えることが多く、それを射と呼ぶ。
つまり、ここで定義するのは順序集合たちのなす圏における射という言い方ができる。

\begin{dfn}
順序集合$X, Y$の間の射$f:X\to Y$とは、$x, x'\in X$で$x\leq x'$であれば$Y$において$f(x)\leq f(x')$となる写像のこと。
つまり、$X$において順序関係を持つ二つの要素を$f$で写すと、$Y$においても順序関係を持っているように定まっている写像である。
\end{dfn}

\begin{eg}
集合$[1]$を通常の大小によって順序集合とみなす。
前に列挙した写像$[1]\to [1]$について、$f_2$以外は順序集合の射であり、
$f_2$は$0\leq 1, f_2(0)=1\geq f_2(1)=0$のため順序集合の射にならない。

一方$[1]$に自明な順序により順序集合とみなす。
すると、全ての写像は順序集合の射二なる。
\end{eg}

この例で見たように、集合の写像$f:X \to Y$について、$X, Y$にどのような順序を定めるかによって順序集合の射になったりならなかったりする。

\begin{eg}
冪集合を上の例のように包含関係により順序集合とみなす。
集合の写像$f:X \to Y$から定まる$P(f):P(Y) \to P(X)$は順序集合の射である。
つまり、$Y$の部分集合$U, V \subset Y$が$U\subset V$をみたすならば$P(f)(U) \subset P(f)(V)$である。
\end{eg}

\subsection{順序集合における米田の補題}
次に紹介するのは、順序集合における米田の補題と言うべきものである。
順序集合における要素は、その順序集合全ての要素との関係により決定されるというもの。

例えばはじめに紹介した$[n]$のような全ての要素が一列に並んでいる場合、ある要素$x$はそれ以下の要素全てを列挙することで$x$を特定することができる。
$[3]$であれば$0$に対して$\{0\}$を対応させ、$1$に対して$\{0,1\}$、$2$に対して$\{0,1,2\}$、$3$に対して$\{0,1,2,3\}$といったようにすればよい。

$P([1])$のような場合でも、$\emptyset$に対して$\{\emptyset\}$、$\{0\}$に対して$\{\emptyset, \{0\}\}$、$\{1\}$に対して$\{\emptyset, \{1\}\}$、$[1]$に対して$P([1])$となるから、この対応により$P([1])$の要素が全て区別できる。

これを一般化するため以下のように考えよう。
$x\in X$に対し関数$h_x:X\to [1]$を$h_x(y)=\Hom_X(y,x)\mbox{の元の個数}$として定める。
つまり
\[
h_x(y)=\begin{cases}1&y\leq x\\0&\mbox{それ以外}\end{cases}
\]
この関数で$1$になるものを全て集めるということが、上で述べたそれ以下の要素を全て列挙するということと同じことになる。

従って、以下のようなことが証明できる。
\begin{prob}
$x\leq y$であることと任意の$z\in X$で$h_x(z)\leq h_y(z)$であることが同値であることを示せ。
特に$x=y$であることと任意の$z\in X$で$h_x(z)=h_y(z)$であることが同値である。

ここで$h_x(z)\leq h_z(y)$は$0\leq0, 0\leq1, 1\leq1$なる順序(つまり通常の大小関係)で定まっている。
\end{prob}

$x\leq y$としよう。
$h_x(z)=0$なら常に$h_x(z)\leq h_y(z)$である。
$h_x(z)=1$なら$z\leq x$であり、$z\leq y$となるので$h_y(z)=1$である。
従って、この場合任意の$z \in X$について$h_x(z)\leq h_y(z)$であることがわかった。

同様にして残りの部分も証明できる。

\subsection{順序集合の図}

順序集合$X$から以下の手順で図を描くことができる。
\begin{enumerate}
\item まず$X$の元に対応する頂点を描く。
\item 次に$x\leq y$のとき$x$から$y$に向かう矢印を描く。
\end{enumerate}

例えば$X=[1]$で自明な順序のみの場合、
\begin{tikzcd}
0 \ar[loop] &  1 \ar[loop]
\end{tikzcd}

$X=[1]$に通常の整数の順序を入れた場合、
\begin{tikzcd}
0 \ar[loop] \ar[r] &  1 \ar[loop]
\end{tikzcd}

$X=P([1])$の場合、
\begin{tikzcd}
& \{0,1\} \ar[loop] & \\
\{0\} \ar[loop] \ar[ur] & &  \{1\} \ar[ul] \ar[loop]\\
& \emptyset \ar[ur] \ar[uu] \ar[ul] \ar[loop]&
\end{tikzcd}

$X=\mathbb{N}_+$の場合、一部を書くと
\begin{tikzcd}
1 \ar[loop] \ar[d] \ar[dr] \ar[drr] \ar[drrr] \ar[dd] \ar[ddr] \ar[ddrr] \ar[ddrrr] \ar[ddd]\\
2 \ar[loop] \ar[d] \ar[dr] \ar[drrr] \ar[dd] &3 \ar[loop] \ar[d] \ar[dr] &5 \ar[loop] \ar[dr] &7 \ar[loop]\\
4 \ar[d] \ar[loop] &6 \ar[loop] &9 \ar[loop] &10 \ar[loop] \\
8 \ar[loop] \\
\end{tikzcd}

のようになる。
これらの絵は「幾つかの点とそれを結ぶ矢印たちの集まりで適当な条件をみたすもの」であり、これは圏の例を与えている。
\begin{itemize}
\item 点たちを圏の対象という。
ここでは対象を集めた集合は$X$である。
\item また二点$x, y$の間の矢印の集合を$\Hom_X(x,y)$とかく。
ここでは
\[
\Hom_X(x,y)=\begin{cases}\{\to\} &x\leq y\\ \emptyset &\mbox{そうでない時}\end{cases}
\]
\end{itemize}

順序集合の定義によれば、
\begin{enumerate}
\item それぞれの点について、始点と終点が自分自身であるような矢印がただ一つある
\item 異なる二点の間の矢印があるとすればどちらか向きに一つだけある
\item 二つの矢印をつないだ矢印がある
\end{enumerate}
という絵を描くことと順序集合を考えることは全く同等のことである。

順序集合の間の射をこの絵を使って考えると、点を点にうつし矢印を矢印にうつすものになる。
これが関手の例を与えている。

\section{圏と関手}

上で描いた絵の条件を緩めて二点を結ぶ矢印がたくさんあってもよいことにし、
一点自身を結ぶ矢印のうち恒等射と呼ばれる特別なものを一つ選ぶことにする。
さらに、二つの矢印を繋げるとどの矢印になるか方も指定する。

そのように点と矢印を描いたものを圏だと思うことができる。
順序集合の場合には、二つの点を結ぶ矢印は多くとも一つしかなかったので恒等射や矢印の繋がり方は必然的に決まったが、
一般には候補がいくつもあるので、単純に絵を描いただけでは圏の情報を全て漏らさず表現することはできないことに注意しよう。

\subsection{圏の定義と例}
上のことを踏まえて以下のように定義する。
\begin{dfn}[圏]
圏$C$とは
\begin{itemize}
\item 対象の集まり$\Ob(C)$
\item 任意の$x, y\in \Ob(C)$にたいし射の集まり$\Hom_C(x,y)$
\end{itemize}
及び、
\begin{itemize}
\item 対象$x\in \Ob(C)$に対し恒等射$\id_x\in\Hom_C(x,x)$
\item 射の合成$\Hom_C(x,y)\times\Hom_C(y,z)\to \Hom_C(x,z); (f,g)\to g\circ f$
\end{itemize}
が定まっていて、全ての射$f, g, h$について
\begin{align*}
\id_x\circ f=f\circ \id_x=f\\
(f\circ g)\circ h=f\circ(g\circ h)
\end{align*}
が成り立つもの。
\end{dfn}
絵で言えば$\Ob(C)$が点の集合、$\Hom_C(x,y)$が$x$から$y$向きの矢印の集合で、矢印は$0$本でもいいし無限にあってもいい。
\footnote{実際には対象全体や射全体が集合でないような圏を扱うこともあるが、ここでは全て圏といえば対象全体や射全体が集合となるようなものとする} 

射の合成というのが、上で述べた矢印の繋ぎ方である。
恒等射とある矢印$f$を繋いだら$f$そのものになり、また三つの矢印をつなぐ時にはそのつなぐ順番によらず決まるということが要請されている。

すでにいくつか紹介しているが、圏の例を見ていこう。

\begin{eg}
圏$\Sets$を
\begin{itemize}
\item 対象の集合$\Ob(\Sets)$を集合全て\footnote{その時点で思いつく集合すべてぐらいの意味で、それ以上深く考えない}を集めたもの
\item 射の集合$\Hom_{\Sets}(x,y)$を集合$x,y$の間の写像全てを集めたもの
\end{itemize}
で定義する。

対象$x$の恒等射は集合$x$の恒等写像$\id_x$とし、射の合成は合成写像$g\circ f(x) = g(f(x))$で定義する。
これらはともに集合の間の写像で、条件を満たす。
\end{eg}

\begin{eg}
順序集合$X$を考える。
これに対して圏$X$を
\begin{itemize}
\item 対象の集合$\Ob(X)$を集合$X$
\item 射の集合$\Hom_{X}(x,y)$を
\begin{align*}
\Hom_X(x,y)=\begin{cases}\{\to\} &x\leq y\\ \emptyset &\mbox{そうでない時}\end{cases}
\end{align*}
\end{itemize}
で定義する。

各対象の間の射は多くとも一つしかないので、射の結合はできるのであれば自動的に決まってしまう。
$x \leq x$だから、$\Hom(x,x)$は空でなく従って$\id_x$が存在する。
また$x \leq y, y \leq z$ならば$x \leq z$だから、射の合成$\Hom(x,y) \times \Hom(y,z) \to \Hom(x,z)$が定義できる。

逆に圏$C$が、すべての射の集合が高々1元でしかも$\Hom_C(x,y)$か$\Hom_C(y,x)$のいずれかは必ず空である、
という条件を満たせば、上の逆の対応で順序集合を作ることができる。

このような圏を順序集合に対応する圏ということにする。
\end{eg}

\begin{eg}
実ベクトル空間全体の圏$\Vect_\R$を
\begin{itemize}
\item 対象の集合$\Ob(\Vect_\R)$を$\R$線形空間全てを集めたもの
\item 射の集合$\Hom_{\Vect_\R}(x,y)$を$x,y$の間の$\R$線形写像全てを集めたもの
\end{itemize}
で定義する。

対象$x$の恒等射はベクトル空間$x$の恒等写像$\id_x$とし、射の合成は合成写像$g\circ f(x) = g(f(x))$で定義する。
これらはともに線形写像であり、条件を満たす。
\end{eg}

\begin{eg}
圏$\Gamma$を
\begin{itemize}
\item 対象の集合$\Ob(\Gamma)=\{e,v\}$
\item 射の集合$\Hom_\Gamma(e,e)=\{\id_e\},\Hom_\Gamma(e,v)=\{s,t\}, \Hom_\Gamma(v,v)=\{\id_v\},\Hom_\Gamma(v,e)=\emptyset$
\end{itemize}
で定義する。これを図で描くと
\begin{xy}
(0,0)*{\bullet}*+!D{0}="A",
(10,0)*{\bullet}*+!D{1}="B",
\ar @(lu,ld) "A";"A"
\ar @(ru,rd) "B";"B"
\ar @<1mm> "A";"B"
\ar @<-1mm> "A";"B"
\end{xy}
となる。
\end{eg}

\begin{rem}
繰り返しになるが、圏を上のように図示しても一般には射の合成の様子までは見えないことに注意する。
例えば
\begin{itemize}
\item $C$を$\Ob(C)=\{x\}, \Hom_C(x,x)=\{id_x,f\}$で射の合成が$f\circ f=id_x$から定まる圏
\item $C'$を$\Ob(C)=\{x\}, \Hom_C(x,x)=\{id_x,f\}$で射の合成が$f\circ f=f$から定まる圏
\end{itemize}
とする。

このとき、どちらも
\begin{xy}
(0,0)*{\bullet}*+!D{0}="x",
\ar @(lu,ld) "x";"x"
\ar @(ru,rd) "x";"x"
\end{xy}
という図になって、図からは区別ができないが、二つの圏は異なるものである。
\end{rem}

与えられた圏$C$に対してその射の向きを全てひっくり返した圏を考えることができる。
\begin{dfn}
圏$C$にたいし、圏$C^{\rm op}$を
\begin{itemize}
\item 対象の集合$\Ob(C^{\rm op})$は$\Ob(C)$と同じ
\item 射の集合$\Hom_{C^{\rm op}}(x,y)$は$\Hom_C(y,x)$と同じ
\end{itemize}
により定義する。
\end{dfn}
これは上で説明したような図で言うと、$C$と同じ点を持ち矢印の向きを逆にした圏ということである。

\begin{rem}
集合の圏のように、実際に射が何らかの写像となっている場合、写像の向きと射の向きが逆になるので混乱しやすい。
つまり、集合$X, Y$とその写像$f:X \to Y$は$\Sets$における射$f\in\Hom_{\Sets}(X,Y)$を定めており、これは同時に$f\in\Hom_{\Sets^{\rm op}}(Y,X)$も定めているが、あくまでも写像としては$X$の要素に対して$Y$の要素を対応させるものである。

この混乱を避けるために、上で決まる$\Sets^{\rm op}$の射は$f^{\rm op}$と書くことにする。
\end{rem}

\begin{eg}
順序集合$X$について、それを圏とみなしたものも$X$と書くことにする。
$X^{\rm op}$は全ての大小関係をひっくり返してできる順序集合を圏とみなしたものである。
\end{eg}

圏においては矢印、つまり射が重要ということがよく言われる。
別の言い方をすると、ある対象と他の対象との関係を表現するのが圏である。

その意味で、他の対象との関係だけでは区別することができない二つの対象のことを同型という。
正確には以下のように定義する。

\begin{dfn}
圏$C$の射$f:x\to y$が同型とは、ある$g:y\to x$があって$f\circ g=\id_x, g\circ f=\id_y$なることを言う。
この時$x$と$y$は同型であるともいう。
\end{dfn}

\begin{eg}
集合の圏$\Sets$において、一元からなる集合は全て同型である。
また、ベクトル空間の圏$\Vect_\R$において同じ次元$n$を持つベクトル空間は全て同型である。
\end{eg}

\subsection{関手の定義と例}
二つの圏を結びつけるために、関手を定義する。
前に説明した順序集合の例で言えば、順序集合$X, Y$を圏とみなしたとき、順序を保つ写像$f:X \to Y$が関手に対応する。

これを一般の圏について定義しよう。

\begin{dfn}[関手]
$C, D$を圏とする。関手$F:C\to D$とは
\begin{itemize}
\item 対象の集合の間の写像$F_{\Ob}:\Ob(C)\to \Ob(D)$
\item 任意の$C$の対象$x,y$について射の集合の間の写像$F_{x,y}:\Hom_C(x,y)\to \Hom_D(F(x),F(y))$
\end{itemize}
なる写像のたちの集まりであって、
\begin{itemize}
\item 任意の対象$x\in\Ob(C)$にたいし$F_{x,x}(\id_x)=\id_{F_{\Ob}(x)}$となる
\item 任意の射$f:x\to y, g:y\to z$に対し、$F_{x,z}(g\circ f)=F_{y,z}(g)\circ F_{x,y}(f)$となる
\end{itemize}
という条件をみたすもの。
\end{dfn}

このような関手を共変関手と呼ぶこともある。

\begin{rem}
関手$F:C\to D$として射の対応を$f:x\to y$の行き先を$F(f):F(y)\to F(x)$とするものを考え、これを反変関手と呼ぶこともある。
これは$C^{\rm op}$からの関手を考えるとみなせるので、ここでは反変関手という言葉は使わない。
\end{rem}

\begin{eg}
圏$C$に対して恒等関手$\id_C$を対象$x$を$x$にうつし、射$f:x \to y$を射$f:x \to y$にうつす関手とする。
\end{eg}

\begin{eg}
冪集合により定まる関手$P:\Sets^{\rm op}\to\Sets$を考える。

これは、
\begin{itemize}
\item 写像$\Ob(\Sets^{\rm op})\to \Ob(\Sets)$を集合$X$に対してその冪集合$P(X)$を対応させる
\item 写像$\Hom_{\Sets^{\rm op}}(Y,X)\to \Hom_{\Sets}(P(Y),P(X))$を写像$f:X\to Y\in\Hom_{\Sets^{\rm op}}(Y,X)$に$P(f):P(Y)\to P(X)\in\Hom_{\Sets}(P(X),P(Y))$を対応させる
\end{itemize}
によって定めることができる。

$A \subset X$に対し、$P(\id)(A)=A$であること、及び集合の写像$f:X \to Y, g:Y \to Z, B\subset Z$に対して
$\Sets^{\rm op}$の射$f^{\rm op}:Y \to X, g^{\rm op}:Z \to Y$が定まり、$B \subset Z$に対して
\begin{align*}
P(f^{\rm op})(P(g^{\rm op})(B))=P(f^{\rm op}\circ g^{\rm op})(B)
\end{align*}
が成り立つ。
\end{eg}

\begin{eg}
順序集合$X, Y$をそれぞれ圏とみなす。
すると$X, Y$の間の順序集合の射$f:X \to Y$とは関手$f:X \to Y$である。

まず$f_{\Ob}$は集合の写像としての$f$とする。
また$f_{x,y}$だが、
\begin{align*}
\Hom_X(x,y)=\begin{cases}\{\to\} &x\leq y\\ \emptyset &\mbox{そうでない時}\end{cases}
\end{align*}
であること、及び空でない集合から空集合への写像は存在しないことに注意すると、
$x\leq y$の時には$\Hom_Y(f(x),f(y))\neq\emptyset$でなくてはならず、これはつまり$f(x)\leq f(y)$ということになる。
つまり、$x\leq y$ならば$f(x)\leq f(y)$ということで、これは順序集合の射の条件そのもの。
\end{eg}

\begin{eg}
集合$X$との直積をとるという操作により関手$\Sets \to \Sets$が定まる。

\begin{itemize}
\item 写像$\Ob(\Sets)\to \Ob(\Sets)$を集合$Y$に対して集合$Y \times X$を対応させる
\item 写像$\Hom_{\Sets}(Y,Z)\to \Hom_{\Sets}(Y\times X,Z \times X)$を写像$f:Y \to Z$に対して、$f\times \id_X(y,x)=(f(y),x)$で定まる写像$f\times \id_X:Y \times X \to Z \times X$を対応させる
\end{itemize}
によって定める。
\end{eg}
上の直積の例では、$Y$に対して$X \times Y$を対応させることでも同じような関手を作ることができる。
このようにして作った関手は異なるもののように思えるが、直積で左右入れ替えるということを考えるとどちらも同じようなものと言える。
この二つの関手は自然に同一視できる。
つまり、後で述べるように上の二つの関手の間には自然変換があり、その意味で同型になる。

\vspace{10pt}
次の例は、上と違って二つの集合から直積を作るという操作を関手とみなす。
その関手の定義域として、二つの集合の組を対象とする圏を考えることになるが、後の話を見据えてこれを関手の圏として記述してみよう。
\begin{eg}
二つの集合の直積を関手として捉えよう。
まず集合$\{0,1\}$を圏とみなす。
これは対象が$\{0,1\}$の二つで、射はそれぞれの恒等射$\id_0, \id_1$のみからなるもの。
この集合に自明な順序のみを定めた順序集合に対応する圏と言ってもよい。

次に関手$I:\{0,1\} \to \Sets$を考える。
これは単に集合二つを選択すると考えればよい。
実際、この関手が定めるべきデータは集合$I(0), I(1)$と射$I(\id_0), I(\id_1)$であるが、後者は$\id_{I(0)}, \id_{I(1)}$にならなければならず、
従って$I$を決めるということは$\Sets$の対象である集合$I(0), I(1)$を選ぶことに他ならない。

このような関手全体のなす圏を考える。
つまり、集合二つの組を対象とし、射は集合の写像二つの組であるような圏を考える。
この圏を$\Sets^2$と書くことにしよう。

直積を取るという操作により関手$\Sets^2 \to \Sets$が定まる。
\begin{itemize}
\item 写像$\Ob(\Sets^2)\to \Ob(\Sets)$を集合の組$(X,Y)$に対して集合$X \times Y$を対応させる
\item 写像$\Hom_{\Sets^2}((X,Y),(Z,W))\to \Hom_{\Sets}(X\times Y,Z \times W)$を写像の組$(f:X \to Y, g:Z \to W)$に対して、$(f \times g)(x,z)=(f(x), g(z))$で定まる写像$f\times g:X \times Y \to Z \times W$を対応させる
\end{itemize}
によって定める。
\end{eg}

\begin{eg}
有向グラフを関手$\Gamma \to \Sets$とみなすことができる。
有向グラフというのはいくつかの頂点とそれらを結ぶいくつかの辺からなる図のこと。
関手$F:\Gamma \to \Sets$を定めることで、
\begin{itemize}
\item $\Gamma$の対象$\{e,v\}$に対して$\Sets$の対象、つまり集合$F(e), F(v)$
\item $\Gamma$の射$s:e \to v, t:e \to v$に対し$\Sets$の射、つまり集合の写像$F(s):F(e) \to F(v), F(t):F(e) \to F(v)$
\end{itemize}
が定まる。

$F(v)$を頂点の集合とし、$F(e)$を辺の集合とする。
各辺$e_i \in F(e)$に対して頂点$F(s)(e_i) \in F(v)$と$F(t)(e_i) \in F(v)$を結ぶ。

例えば$F$を$F(v)=\{a,b,c,d\}, F(e)=\{1,2,3,4,5,6\}$とし、$F(s), F(t)$を
\begin{align*}
(F(s)(1), F(s)(2), F(s)(3), F(s)(4), F(s)(5), F(s)(6)) = (a,a,a,b,c,d)\\
(F(t)(1), F(t)(2), F(t)(3), F(t)(4), F(t)(5), F(t)(6)) = (b,c,d,c,d,b)
\end{align*} 
とすると、以下のような有向グラフが書ける。

\begin{center}
\begin{xy}
(0,0)*{\bullet}*++ !D{a}="a",
(0,10)*{\bullet}*++!D{b}="b",
(8.5,-5)*{\bullet}*++!D{c}="c",
(-8.5,-5)*{\bullet}*++!D{d}="d",
\ar "a";"b"
\ar "a";"c"
\ar "a";"d"
\ar "b";"c"
\ar "c";"d"
\ar "d";"b"
\end{xy}
\end{center}
\end{eg}

次の例はベクトル空間の定義を知っている人向けのもので、特にこの先必要なものではないので読み飛ばしてもらって構わない。
\begin{eg}
集合の圏$\Sets$から実ベクトル空間の圏$\Vect_\R$への関手として
\begin{itemize}
\item 集合$X$に対して$X$を定義域にもつ実数値関数全体$\R^X$を対応させる関手$\Sets \to \Vect_\R$
\item 集合$X$に対して$X$を基底とするベクトル空間$\R^{\oplus X}$を対応させる関手$\Sets \to \Vect_\R$
\item 集合$X$に対して$X$の要素を変数にもつ多項式全体$\R[X]$を対応させる関手$\Sets \to \Vect_\R$
\end{itemize}
を定義することができる。
これが関手となるためにはどのように射を対応させればよいか考えてみよ。

またベクトル空間$V$からその双対空間$\Hom_\R(V, \R)$をとることで関手$\Vect_\R \to \Vect_\R$が定まる。
\end{eg}

\subsection{表現可能関手}
様々な関手の中で、表現可能関手と呼ばれる特によい性質を持つ関手について説明したい。
これは後で米田の補題を述べるためにも必要である。

圏$C$の対象$x$に対して定まる二種類の関手$g_x:C \to \Sets$と$h_x:C^{\rm op} \to \Sets$を以下のようにして定める。

\begin{dfn}
$x$を圏$C$の対象とする。
関手$g_x:C\to \Sets$を
\begin{itemize}
\item $C$の対象$y\in\Ob(C)$にたいし集合$\Hom_C(x,y)\in\Ob(\Sets)$
\item $C$の射$f:y\to z$にたいし、集合の写像$g_x(f)=f^*:\Hom_C(x,y)\to\Hom_C(x,z)$を$\phi\mapsto f\circ\phi$
\end{itemize}
として定義する。
\end{dfn}

\begin{eg}
$C=\Sets$の対象$x=[0]$が定める関手$g_{[0]}:\Sets \to \Sets$を$X \mapsto \Hom([0], X)$により定める。
これは前にも述べたように、実際には$X$そのものと自然に同一視できる。
\end{eg}

\begin{eg}
$C=\Sets$の対象$x=[1]$が定める関手$g_{[1]}:\Sets \to \Sets$を$X \mapsto \Hom([1], X)$により定める。
これは、実際には$X$の元二つを順序付きで並べたものの集まりと自然に同一視できる。
つまり、$f:[1] \to X$という写像は$f(0)=x, f(1)=y$なら、$(x,y)$という組を指定することと同じ。

さらに、この関手による射の対応は、$g:X \to Y$に対して$g_{[1]}(g):\Hom([1], X) \to \Hom([1], Y)$を$(x,y)$という組みに対応する写像を$(g(x), g(y))$という組みに対応する写像に写すもの。
\end{eg}

\begin{dfn}
$x$を圏$C$の対象とする。
関手$h_x:C^{\rm op}\to \Sets$を
\begin{itemize}
\item $C$の対象$y\in\Ob(\Sets)$にたいし集合$\Hom_C(y,x)\in\Ob(\Sets)$
\item $C$の射$f:y\to z$にたいし、集合の写像$h_x(f)=f^*:\Hom(y,x)\to\Hom(z,x)$を$\phi\mapsto \phi\circ f$
\end{itemize}
として定義する。
\end{dfn}

\begin{eg}
$C=\Sets$の対象$x=[1]$が定める関手$h_{[1]}:\Hom(X,[1])$を$X \mapsto \Hom(X,[1])$により定める。
写像$f:X \to [1]$を決めるということは$X$の要素を$0$にうつる要素と$1$にうつる要素の二つに分けるということ。
つまり$X$を二つの集合$A, B$に分けるということに対応する。

このように見たとき、関手による射の対応はどのように見えるか。
写像$g:X \to Y$に対して$g^{\rm op}:Y \to X$が定まり、$h_{[1]}(g^{\rm op}):\Hom(Y,[1]) \to \Hom(X,[1])$が定まる。
$Y$を二つの集合$C, D$に分けることで決まる写像$f:Y \to [1]$を$h_{[1]}(g^{\rm op})$でうつすと、$h_{[1]}(g^{\rm op})(f)=f\circ g:X \to [1]$という写像になる。
これは$X$の二つの集合$A, B$として$g$により$C$にうつるもの全体、$D$にうつるもの全体、の二つに分けることになる。
\end{eg}

\begin{eg}
$C=\Sets$の対象$x=[1]\times[1]$が定める関手$h_{[1]\times[1]}\Hom(X,[1] \times [1])$を$X \mapsto \Hom(X,[1]\times[1])$により定める。
写像$f:X \to [1] \times [1]$というのは$f(x)=(0,0), (0,1), (1,0), (1,1)$のいずれかであり、これは$X$を四つの集合に分けることになる。

しかし、これは単に四つに分けるというよりは、$[1] \times [1]$が直積であるということを考えると、$X$を二つの集合に分ける方法$A, B$と$C,D$の二通りを指定すると考える方がよい。

このように見たときに、射の対応がどのようになるか、上の例と同様に考えよう。
\end{eg}

\begin{eg}
順序集合$X$とその要素$x \in X$から定めた$h_x$という関数は表現可能関手の例である。
その定義を復習すると、$x\in X$に対し関手$h_x:X\to \Sets$を$h_x(y)=\Hom_X(y,x)\mbox{の元の個数}$として定める。
つまり
\[
h_x(y)=\begin{cases}1&y\leq x\\0&\mbox{それ以外}\end{cases}
\]

関手として改めてこの関数を記述してみよう。
圏$X$の対象$x\in X$に対し関手$h_x:X\to \Sets$を$h_x(y)=\Hom_X(y,x)$として定める。
つまり
\[
h_x(y)=\begin{cases}\{\to\}&y\leq x\\\emptyset&\mbox{それ以外}\end{cases}
\]
順序集合の定義で$x \leq x$と$x\leq y, y\leq z$ならば$x\leq z$であることから、これが関手になることがわかる。

下の関手からその行き先の要素の個数を数えることで、上の関数を定めることができる。
\end{eg}

\section{自然変換の定義と例}
ここまでいくつかの例において、\emph{自然な}対応、\emph{自然な}写像といったものがでてきた。
改めてその例を見ながら、\emph{自然変換}の定義を確認していく。

実は自然変換は関手の圏の射とみなすことができる。
二つの圏$C, D$に対して、その間の関手全体の圏$C^D$を考えることができる。
この関手の圏の射$\phi:F\to G$こそが自然変換である。
この節の終わりに、このようにして自然変換を定義する。

では、まずはここまでに出てきた「自然な」対応の例を振り返ってみよう。

\subsection{一点集合からの写像}
集合の圏$\Sets$から$\Sets$への関手として、$\Sets$の対象$[0]=\{0\}$により定まるもの$g_{[0]}$があった。
これは、
\begin{itemize}
\item 対象の対応が$X \mapsto \Hom_{\Sets}([0], X)$
\item 射の対応が$(f:X \to Y) \mapsto (f_*:\Hom_{\Sets}([0],X) \to \Hom_{\Sets}([0],Y)$
\end{itemize}
として定まる関手である。

ここで集合$\Hom([0],X)$がどのようなものか考えてみよう。
$f\in\Hom([0],X)$は一点集合$[0]=\{0\}$から$X$への写像なので、$0$のうつり先$f(0)$のみで決定される。
したがって、$\Hom([0],X)$と$X$は$f$に対して$f(0)$を対応させることで、集合として同一視することができる。
つまり、全単射$\phi(X):\Hom([0],X) \to X$が$\phi(X)(f)=f(0)$により定まる。

この同一視は単に要素がぴったり対応するという以上に、\textbf{自然な}同一視と呼ばれる特別な同一視である。
これがどういうことかを見ていく。

上の写像を$\phi(X)$という形であえて$X$を明示したが、これはなぜかというと別の集合$Y$についても同様に
全単射$\phi(Y):\Hom([0],Y) \to Y$が$\phi(Y)(f)=f(0)$により定まるからである。

さらに、写像$g:X \to Y$が存在したとすると、$g_{[0]}$が関手であるから$g_{[0]}(g):g_{[0]}(X) \to g_{[0]}(Y)$が定まる。
この$g_{[0]}(g)$は$f:[0] \to X$に$g$を合成して$g\circ f:[0] \to Y$を対応させるという写像であった。
この$g_{[0]}(g)(f)\in g_{[0]}(Y)=\Hom([0],Y)$を$\phi(Y)$を用いて$Y$にうつすと、$\phi(Y)(g_{[0]}(g)(f))=g_{[0]}(f)(0)=(g\circ f)(0)=g(f(0))=g(\phi(X)(f))$となる。
つまり、
\begin{align*}
\phi(Y)(g_{[0]}(g)(f))=g(\phi(X)(f))
\end{align*}
である。
左辺は$f\in g_{[0]}(X)=\Hom([0],X)$を$g_{[0]}(g):g_{[0]}(X) \to g_{[0]}(Y)$でうつし、さらに$\phi(Y):g_{[0]}(Y) \to Y$でうつしたもので、
右辺は$f\in g_{[0]}(X)=\Hom([0],X)$を$\phi(X):g_{[0]}(X) \to X$でうつし、さらに$g:X \to Y$でうつしたものである。
上の等式はこの二つが一致するということをいっており、このような状況を表すために\emph{可換図式}というものを導入する。

今出てきた四つの集合とその間の四つの写像を下のような図に表す(この図の上下左右には特に意味はない)。
\[
\begin{CD}
g_{[0]}(X)@>\phi(X)>>X\\
@Vg_{[0]}(g)VV@VgVV\\
g_{[0]}(Y)@>\phi(Y)>>Y
\end{CD}
\]
この図式が可換であるとは、$g_{[0]}(X)$から$Y$に向かうふた通りの方法で写像を合成して得られる二つの写像
$\phi(Y)\circ g_{[0]}(g)$と$g\circ \phi(X)$が一致する、言い換えると$f\in g_{[0]}(X)$に対して
\begin{align*}
\phi(Y)(g_{[0]}(g)(f))=g(\phi(X)(f))
\end{align*}
が成り立つということをいう。

これをふまえて、改めて$g_{[0]}(X)$と$X$が自然に同一視できるということを述べる。
全ての集合$X, Y$及びその間の写像$g:X \to Y$に対して、全単射$\phi(X):g_{[0]}(X) \to X$が存在して、以下の図式
\[
\begin{CD}
g_{[0]}(X)@>\phi(X)>>X\\
@Vg_{[0]}(g)VV@VgVV\\
g_{[0]}(Y)@>\phi(Y)>>Y
\end{CD}
\]
が可換であるとき、$g_{[0]}(X)$と$X$が自然に同一視できるという。

\vspace{10pt}
上の写像は全単射であったから逆写像が存在するので、それを明示的に述べよう。
まず集合$X$に対して、$\psi(X):X \to g_{[0]}(X)=\Hom([0],X)$を定める。
$X$の各要素$x\in X$に対して、$\psi(X)(x):[0] \to X$を$0$の行き先が$x$であるような写像として定義する。
同様にして、集合$Y$に対しても$\psi(Y):Y \to g_{[0]}(Y)$も定めることができる。

写像$g:X \to Y$に対して、次の図式を可換にすることを確かめよう。

\[
\begin{CD}
X@>\psi(X)>>g_{[0]}(X)\\
@VgVV@Vg_{[0]}(g)VV\\
Y@>\psi(Y)>>g_{[0]}(Y)
\end{CD}
\]
上の図式が可換であるとは$X$から$g_{[0]}(Y)$への二つの写像$\psi(Y)\circ g$と$g_{[0]}(g)\circ \psi(X)$が一致するということであり、
いいかえると$x\in X$に対して$g_{[0]}(g)(\psi(X)(x))=\psi(Y)(g(x))$がなりたつということを確かめればよい。

$\psi(X)(x)$は$0\in[0]$を$x\in X$にうつす写像$[0] \to X$であり、これを$g_{[0]}(g)$でうつすというのは$g$を合成するということ。
つまり、$g_{[0]}(g)(\psi(X)(x))$は$0$を$g(x)$にうつす写像$[0] \to Y$である。

一方$\psi(Y)(g(x))$は$0 \in [0]$を$g(x) \in Y$にうつす写像であるから、$g_{[0]}(g)(\psi(X)(x))$と$\psi(Y)(g(x))$はともに写像$[0] \to Y$で$0$を$g(x)$にうつす写像であり、この二つは一致することがわかる。
したがって確かに図式は可換になり、$\psi(X)$たちは自然な同一視を定めている。

\vspace{10pt}
上で述べたことを二つの関手の間の対応という形でいいかえる。
恒等関手$\id:\Sets\to \Sets$は$\id(X)=X, \id(f)=f$で定まる関手である。
$\Sets$の各対象$X$に対して、写像$\phi(X):g_{[0]}(X)=\Hom([0],X) \to \id(X)=X$を$f$に対して$f(0)$を対応させることで定める。
すると、$\Sets$の任意の射$g:X \to Y$に対して、次の図式
\[
\begin{CD}
g_{[0]}(X)@>\phi(X)>>\id(X)\\
@Vg_{[0]}(g)VV@V\id(g)VV\\
g_{[0]}(Y)@>\phi(Y)>>\id(Y)
\end{CD}
\]
が可換になる。
このように$\phi(X)$の集まりであって上の図式を可換にするものを、二つの関手$g_{[0]}$と$\id$の間の\textbf{自然変換}$\phi:g_{[0]} \to \id$という。

これと同様にして、上で説明したように自然変換$\psi:\id \to g_{[0]}$を定めることができ、この二つの自然変換は互いに逆の対応を与えている。

\vspace{10pt}

自然変換の例をもう一つ見てみよう。
$C=\Sets$の対象$x=[1]$が定める関手$g_{[1]}:\Sets \to \Sets$を
\begin{itemize}
\item 対象の対応が$X \mapsto \Hom([1], X)$
\item 射の対応が$(f:X \to Y) \mapsto (f_*:\Hom_{\Sets}([1],X) \to \Hom_{\Sets}([1],Y))$
\end{itemize}
により定義する。

$\Hom_{\Sets}([1],X)$の要素$f:[1] \to X$は$f(0), f(1)$のみから一通りに定まる写像で、$X$の元二つを順序付きで並べたものと対応する。
つまり全単射$\phi(X):\Hom_{\Sets}([1],X) \to X \times X$が存在する。
関手$\Delta:\Sets \to \Sets$を
\begin{itemize}
\item 対象の対応が$X \mapsto X \times X$
\item 射の対応が$(f:X \to Y) \mapsto (f\times f:X \times X \to Y\times Y;(x_1,x_2) \mapsto (f(x_1),f(x_2)))$
\end{itemize}
により定義したとき、上の$\phi(X)$が関手$g_{[1]}:\Sets \to \Sets$と$\Delta:\Sets \to \Sets$の間の自然変換$g_{[1]} \to \Delta$を与えることを上と同様に確かめてみよう。

$g:X \to Y$に対して、次の図式。
\[
\begin{CD}
g_{[1]}(X)@>\phi(X)>>\Delta(X)\\
@Vg_{[1]}(g)VV@V\Delta(g)VV\\
g_{[1]}(Y)@>\phi(Y)>>\Delta(Y)
\end{CD}
\]
が可換であればよいので、それを確かめる。
つまり、全ての$f \in g_{[1]}(X)$に対して、
\begin{align*}
\Delta(g)(\phi(X)(f))=\phi(Y)(g_{[1]}(g)(f))
\end{align*}であればよい。
右上を通って$\Delta(Y)$にうつした$\Delta(g)(\phi(X)(f))$を計算すると
\begin{align*}
\phi(X)(f)=(f(0), f(1))\in\Delta(X)=X\times X
\end{align*}
であり、
\begin{align*}
\Delta(g)(f(0),f(1))=(g(f(0)),g(f(1)))
\end{align*}
である。
一方で、左下を通って$\Delta(Y)$にうつした$\phi(Y)(g_{[1]}(g)(f))$を計算すると、
\begin{align*}
\phi(Y)(g_{[1]}(g)(f))= (g_{[1]}(g)(f)(0),g_{[1]}(g)(f)(1))=((g\circ f)(0), (g\circ f)(1))=(g(f(0)), g(f(1)))
\end{align*}
となる。
つまり、上の図式が可換であることがわかり、これにより$\phi(X)$たちが自然変換$g_{[1]} \to \Delta$を与えていることがわかった。

この逆の自然変換$\Delta \to g_{[1]}$についても同様に与えることができるので、考えてみてほしい。

\subsection{直積}
二つの集合$X, Y$の直積$X \times Y$について、\emph{自然に}二つの写像$\pr_1:X \times Y \to X, \pr_2:X \times Y \to Y$が定まる。
これらはそれぞれ対の左側の要素、右側の要素を取り出す写像である。

また$X \times Y$と$Y \times X$は左右の成分を入れ替えることで\emph{自然に}同一視できる。
これらのことについて、以下でふた通りの方法で見てみよう。

\vspace{10pt}

$\Sets$の対象、すなわち集合$X$を固定して考える。
$X$との直積を取る関手$L_X:Y \mapsto X \times Y$と恒等関手$\id:Y \mapsto Y$はいずれも関手$\Sets \to \Sets$を与える。
$\Sets$の各対象$Y$に対して$\pr_2(Y):L_X(Y) \to \id(Y)$という写像を$(x,y) \in L_X(Y)=X\times Y$に対して$\pr_2(Y)(x,y)=y$として定める。
すると、$\Sets$の射$g:Y \to Z$に対して次の図式が可換になる。
\[
\begin{CD}
L_X(Y)@>\pr_1(X)>>\id(Y)\\
@VL_X(f)VV@V\id(f)VV\\
L_X(Z)@>\pr_1(Z)>>\id(Z)
\end{CD}
\]
この図式が可換になることを確かめてみよう。
$(x,y) \in L_X(Y)=X\times Y$をとる。
まず右上を通ると、
\begin{align*}
\id(f)(\pr_2(X)(x,y))=\id(f)(y)=f(y)
\end{align*}
であり、左下を通ると
\begin{align*}
\pr_2(Z)(L_X(f)(x,y))=\pr_2(Z)(x,f(y))=f(y)
\end{align*}
となるので、二つの写像の合成は一致し、図式は可換である。
したがって$\pr_2(Y)$たちが自然変換$\pr_2:L_X \to \id$を与える。
これが自然に$\pr_2:X \times Y \to Y$が定まる、ということの一つの解釈である。

また、$X$に対して、$X$との直積を取る関手$R_X:Y \mapsto Y \times X$と恒等関手$\id:Y \mapsto Y$はいずれも関手$\Sets \to \Sets$を与える。
この二つの間には$\pr_1(Y):L_X(Y) \to \id(Y)$という写像が存在し、次の図式が可換になることが上と同様にわかる。
\[
\begin{CD}
F(x)@>\phi(x)>>G(x)\\
@VF(f)VV@VG(f)VV\\
F(y)@>\phi(y)>>G(y)
\end{CD}
\]
したがって$\pr_1(Y)$たちが自然変換$\pr_1:L_X \to \id$を与える。
これが自然に$\pr_1:X \times Y \to Y$が定まる、ということの一つの解釈である。

最後に$X \times Y$と$Y \times X$が自然に同一視できるということについても、同様に自然変換を用いた解釈を与えよう。
$\Sets$の対象$X$を固定し、上と同様に$X$と直積をとる関手$L_X:Y \mapsto X \times Y$と$R_X:Y \mapsto Y \times X$を考える。
これは$\Sets$から$\Sets$への関手である。

$\Sets$の対象$Y$に対して、
$L_X(Y)=X\times Y$と$R_X(Y)=Y \times X$の間に写像$t(Y):L_X(Y)=X \times Y \to R_X(Y)=Y \times X$を$(x,y) \mapsto (y,x)$により定める。
このとき、$f:Y\to Z$対して次が可換となることを確かめよう。
\[
\begin{CD}
L_X(Y)@>t(Y)>>R_X(Y)\\
@VL_X(f)VV@VR_X(f)VV\\
L_X(Z)@>t(Z)>>R_X(Z)
\end{CD}
\]
$(x,y) \in L_X(Y)=X\times Y$をとる。
まず右上を通ると、
\begin{align*}
R_X(f)(r(Y)(x,y))=R_X(f)(y,x)=(f(y),x)
\end{align*}
であり、左下を通ると
\begin{align*}
t(Z)(L_X(f)(x,y))=t(Z)(x,f(y))=(f(y),x)
\end{align*}
となるので、二つの写像の合成は一致し、図式は可換である。
したがって$t(Y)$たちが自然変換$t:L_X \to R_X$を与えることがわかった。
これの逆の自然変換$R_X \to L_X$も同様に構成でき、$X \times Y$と$Y \times X$が自然に同一視できるということが自然変換を用いて説明できる。

\vspace{10pt}
上の説明では集合$X$を一つ固定していることに不満がある。
実際にはこの$X$もいろいろ取り替えて直積を考えることになるので、それも捉えられるように別の定式化を与えよう。

集合二つ組のなす圏を$\Sets^{[1]}$と書くことにする。
つまり、
\begin{itemize}
\item 対象の集合$\Ob(\Sets^{[1]})$を集合二つ組$(X_0,X_1)$全てを集めたもの
\item 射の集合$\Hom_{\Sets^{[1]}}((X_0,X_1),(Y_0,Y_1))$を写像二つ組$(f_0:X_0 \to Y_0, f_1:X_1\to Y_1)$全てを集めたもの
\end{itemize}
として定義する。
射の合成や恒等射は集合の射の合成や恒等射から定まるものとする。

集合の二つ組からその直積をとるという関手$\times:\Sets^{[1]} \to \Sets$を
\begin{itemize}
\item 対象の対応を集合二つ組$(X_0,X_1)$に対してその直積$X_0 \times X_1$を対応させる、つまり$\times(X_0,X_1)=X_0\times X_1$
\item 射の対応を写像の組$(f_0,f_1):(X_0,X_1) \to (Y_0,Y_1)$に対して、$f_0\times f_1(x_0,x_1)=(f_0(x_0),f_1(x_1))$で定まる写像$f_0\times f_1:X_0 \times X_1 \to Y_0 \times Y_1$を対応させる、つまり$\times(f_0,f_1)=f_0\times f_1$
\end{itemize}
により定める。

集合の二つ組から一つ目の集合をとるという関手$\pr_1:\Sets^{[1]} \to \Sets$を
\begin{itemize}
\item 対象の対応を集合二つ組$(X_0,X_1)$に対して一つ目の集合$X_0$を対応させる、つまり$\pr_1(X_0,X_1)=X_0$
\item 射の対応を写像の組$(f_0,f_1):(X_0,X_1) \to (Y_0,Y_1)$に対して、一つ目の写像$f_0:X_0\to Y_0$を対応させる、つまり$\pr_1(f_0,f_1)=f_0$
\end{itemize}
により定める。

集合の二つ組から二つ目の集合をとるという関手$\pr_2:\Sets^{[1]} \to \Sets$を
\begin{itemize}
\item 対象の対応を集合二つ組$(X_0,X_1)$に対して二つ目の集合$X_1$を対応させる、つまり$\pr_2(X_0,X_1)=X_1$
\item 射の対応を写像の組$(f_0,f_1):(X_0,X_1) \to (Y_0,Y_1)$に対して、二つ目の写像$f_1:X_1\to Y_1$を対応させる、つまり$\pr_2(f_0,f_1)=f_1$
\end{itemize}
により定める。

集合の二つ組の順番を入れ替えるという関手$tr:\Sets^{[1]} \to \Sets^{[1]}$を
\begin{itemize}
\item 対象の対応を集合二つ組$(X_0,X_1)$に対して集合の二つ組$(X_1,X_0)$を対応させる、つまり$tr(X_0,X_1)=(X_1,X_0)$
\item 射の対応を写像の組$(f_0,f_1):(X_0,X_1) \to (Y_0,Y_1)$に対して、写像の組$(f_1,f_0)$を対応させる、つまり$tr(f_0,f_1)=(f_0,f_1)$
\end{itemize}
により定める。

これらの関手の間に以下のようにして自然変換を定めることができる。
$\Sets^{[1]}$の対象$(X_0, X_1)$に対して、写像$\pr_1(X_0,X_1):X_0 \times X_1 \to X_0$を$(x_0, x_1) \mapsto x_0$とすることで、自然変換$\pr_1:\times \to \pr_1$を定める

$\Sets^{[1]}$の対象$(X_0, X_1)$に対して、写像$\pr_2(X_0,X_1):X_0 \times X_1 \to X_1$を$(x_0, x_1) \mapsto x_1$とすることで、自然変換$\pr_2:\times \to \pr_2$を定める。

また$tr$と$\times$を合成して得られる関手$\Sets^{[1]} \to \Sets$と$\times$の間の自然変換$t:\times \to \times \circ tr$を、
$\Sets^{[1]}$の対象$(X_0, X_1)$に対して、写像$tr(X_0,X_1):X_0 \times X_1 \to X_1 \times X_0$を$(x_0, x_1) \mapsto (x_1,x_0)$とすることで定める。

これらが確かに自然変換になっているということ、つまり次の形の図式
\[
\begin{CD}
F(x)@>\phi(x)>>G(x)\\
@VF(f)VV@VG(f)VV\\
F(y)@>\phi(y)>>G(y)
\end{CD}
\]
を可換にすることは前半で述べたのと同様に示すことができる。


\subsection{グラフとその間の射}
有向グラフは関手$\Gamma \to \Sets$であることは前に述べたが、ここでは有向グラフの間の射が自然変換であるということをみる。

$\Gamma$を以前定義した圏
\[
\begin{tikzcd}
e \ar[r,shift left, "s"] \ar[r, shift right, "t"'] & v
\end{tikzcd}
\]
とする。

関手$F\colon\Gamma\to \Sets$は有向グラフに対応するものであることは以前にみた。
関手$F$が定めるデータは
\begin{itemize}
\item 集合$F(e), F(v)$
\item 写像$F(s)\colon F(e)\to F(v), F(t)\colon F(e)\to F(v)$
\end{itemize}
である。

次に有向グラフの間の射はどのようなものであるか。
これは頂点を頂点にうつし、辺を辺にうつし、各辺の始点と終点はうつり先でも始点と終点になっているものである。

たとえば、

つまり、関手$F_1:\Gamma \to \Sets$と$F_2:\Gamma \to \Sets$の間の射$\phi:F_1 \to F_2$とは、
\begin{itemize}
\item 点の間の写像$\phi(v):F_1(v) \to F_2(v)$
\item 辺の間の写像$\phi(e):F_1(e) \to F_2(e)$
\end{itemize}
の組であり、
\begin{itemize}
\item 辺$x \in F_1(e)$をうつした$\phi(e)(x)$の始点$F_2(s)(\phi(e)(x))$が、$x$の始点$F_1(s)(x)$を$\phi(v)$でうつした$\phi(v)(F_1(s)(x))$と一致
\item 辺$x \in F_1(e)$をうつした$\phi(e)(x)$の終点$F_2(t)(\phi(e)(x))$が、$x$の終点$F_1(t)(x)$を$\phi(v)$でうつした$\phi(v)(F_1(t)(x))$と一致
\end{itemize}
つまり、
\begin{align*}
F_2(s)(\phi(e)(x))=\phi(v)(F_1(s)(x))\\
F_2(t)(\phi(e)(x))=\phi(v)(F_1(t)(x))
\end{align*}
となるものである。

これを言い換えると、以下の図式が可換になるような射の組み$\phi(v), \phi(e)$である。
\begin{multicols}{2}
\begin{align*}
\begin{CD}
F_1(e)@>\phi(e)>>F_2(e)\\
@VF_1(s)VV@VF_2(s)VV\\
F_1(v)@>\phi(v)>>F_2(v)
\end{CD}
\end{align*}

\begin{align*}
\begin{CD}
F_1(e)@>\phi(e)>>F_2(e)\\
@VF_1(t)VV@VF_2(t)VV\\
F_1(v)@>\phi(v)>>F_2(v)
\end{CD}
\end{align*}
\end{multicols}
つまり、$\phi$は自然変換$F_1 \to F_2$を与えている。

\subsection{冪集合}
ここでは冪集合により定まる関手$P:\Sets^{\rm op}\to\Sets$を考える。
この関手と$[1]$が定める関手$h_{[1]}$が自然に同一視できること、また補集合や合併、共通部分といった操作が自然変換として解釈できることを見ていく。

\vspace{10pt}

まず関手$P:\Sets^{\rm op}\to\Sets$について復習する。
これは、
\begin{itemize}
\item 対象の対応を集合$X$に対してその冪集合$P(X)$を対応させる
\item 射の対応を写像$f:X\to Y\in\Hom_{\Sets^{\rm op}}(Y,X)$に$P(f):P(Y)\to P(X)\in\Hom_{\Sets}(P(X),P(Y))$を対応させる
\end{itemize}
によって定まるものであった。

$\Sets$の対象$[1]=\{0,1\}$が定める関手$h_{[1]}:\Hom(X,[1])$とは、
\begin{itemize}
\item 対象の対応を$X \mapsto \Hom(X,[1])$に対応させる
\item 射の対応を$f:X \to Y$に対して、$f$を合成するという写像$f^*:\Hom(Y,[1]) \to \Hom(X,[1]);g \mapsto f\circ g$に対応させる
\end{itemize}
によって定まるものだった。

この二つの関手の間の自然変換$\phi:P \to h_{[1]}$と$\psi:h_{[1]} \to P$を以下で定義しよう。

\vspace{5pt}

まず$X$の部分集合$A \subset X$から写像$\chi_A:X \to [1]$を
\begin{align*}
\chi_A(x)=\begin{cases} 1 & x \in A\\ 0 & x \notin A\end{cases}
\end{align*}
で定める。このような写像を$A$の特性関数という。
写像$\phi(X):P(X) \to h_{[1]}(X)$を部分集合$A \subset X$に対してその特性関数$\chi_A:X \to [1]$を対応させることにより定める。
このとき、$\Sets$の射$f:Y \to X$に対し図式
\begin{align*}
\begin{CD}
P(Y)@>\phi(Y)>>h_{[1]}(Y)\\
@VP(f)VV@Vh_{[1]}(f)VV\\
P(X)@>\phi(X)>>h_{[1]}(X)
\end{CD}
\end{align*}
が可換になることを確かめよう。

$A \in P(Y)$すなわち部分集合$A\subset Y$をとる。
右上を通って$P(X)$にうつすと、まず$\phi(Y)(A)$は特性関数$\chi_A:Y \to [1]$であり、
これに対して$h_{[1]}(f)(\chi_A)=\chi_A\circ f:X \to [1]$は
\begin{align*}
\chi_A\circ f(x)=\begin{cases} 1 & f(x) \in A\\ 0 & f(x) \notin A\end{cases}
\end{align*}
で定まる写像である。

一方で左下を通って$P(X)$にうつすと、まず$P(f)(A)=f^{-1}(A)=\{x\in X\mid f(x) \in A\}$であり、
$\phi(X)(P(f)(A))$は$P(f)(A)$の特性関数
\begin{align*}
\chi_{P(f)(A)}(x)=\begin{cases} 1 & x \in P(f)(A)\\ 0 & x \notin P(f)(A)\end{cases}
\end{align*}
である。

$P(f)(A)$の定義から$x\in P(f)(A)$であることと$f(x) \in A$であることが同値なため、二つの写像$\chi_A \circ f$と$\chi_{P(f)(A)}$は一致し
、上の図式が可換であることがわかる。
\vspace{5pt}

逆に$\psi(X):h_{[1]}(X) \to P(X)$を次のようにして定める。
写像$f:X \to [1]$を決めるということは$X$の要素を$0$にうつる要素と$1$にうつる要素の二つに分けるということである。
そこで写像$f:X \to [1]$に対して部分集合を$1$にうつる要素全体$f^{-1}(1)=\{x \in X \mid f(x)=1\} \subset X$を対応させることにより、写像$\psi(X):h_{[1]}(X) \to P(X)$を定めよう。
つまり$\psi(X)(f)=f^{-1}(1)$としよう。
このとき、$\Sets$の射$f:Y \to X$に対して図式
\begin{align*}
\begin{CD}
h_{[1]}(Y)@>\psi(Y)>>P(Y)\\
@Vh_{[1]}(f)VV@VP(f)VV\\
h_{[1]}(X)@>\psi(X)>>P(X)
\end{CD}
\end{align*}
が可換になることを確かめれば、これが自然変換を与えることがわかる。

$g \in h_{[1]}(Y)$すなわち写像$g:Y \to [1]$をとる。
右上を通って$P(X)$にうつすと、まず$\psi(Y)(g)=g^{-1}(1)$であり、
\begin{align*}
P(f)(g^{-1}(1))=f^{-1}(g^{-1}(1))=\{x \in X \mid f(x) \in g^{-1}(1)\}=\{x \in X \mid g(f(x))=1\} \in P(X)
\end{align*}
となる。

一方で左下を通って$P(X)$にうつすと、
$h_{[1]}(f)(g)=g\circ f:X \to [1]$であり、
\begin{align*}
\psi(X)(g\circ f)=(g\circ f)^{-1}(1)=\{x \in X \mid (g\circ f)(x) = 1\}
\end{align*}
となることから、二つの部分集合が一致し、図式が可換であることがわかる。

\vspace{5pt}

以上のことから、これらが二つの関手の間の自然変換を与え、しかも互いに逆になっていることがわかる。

\vspace{10pt}
集合$X$に対して、その部分集合の補集合をとる操作は写像$c(X):P(X)\to P(X)$を定める。
これは実は関手$P$から$P$への自然変換$c:P \to P$を与えることがわかる。
このとき、$\Sets$の射$f:X \to Y$に対して以下の図式
\begin{align*}
\begin{CD}
P(Y)@>c(Y)>>P(Y)\\
@VP(f)VV@VP(f)VV\\
P(X)@>c(X)>>P(X)
\end{CD}
\end{align*}
が可換であることが確かめよう。

$A \in P(Y)$すなわち部分集合$A\subset Y$をとる。
右上を通って$P(X)$にうつすと、まず$c(Y)(A)=Y \setminus A$であり、
これに対して$P(f)(Y\setminus A)=\{x \in X \mid f(x) \in Y \setminus A\}$である。

一方で左下を通って$P(X)$にうつすと、まず$P(f)(A)=\{x \in X \mid f(x) \in A\}$であり、
$c(X)(P(f)(A))=X \setminus P(f)(A)=\{x \in X \mid x \notin P(f)(A)\}=\{x \in X \mid f(x) \notin A\} $である。
このことから、二つの部分集合が一致し、図式が可換であることがわかる。

\vspace{10pt}
先に述べたように、$P$と$h_{[1]}$は同一視できる。
この同一視により、補集合を取るという自然変換$c:P \to P$を自然変換$c:h_{[1]} \to h_{[1]}$と見ることができる。
実は$0$と$1$を入れ替える写像$i:[1] \to [1]$から定まる自然変換$h_i:h_{[1]} \to h_{[1]}$こそが、上の自然変換$c:h_{[1]} \to h_{[1]}$であることがわかる。
このことについては、次回の米田の補題に関連してより詳しく説明します。


\subsection{関手の圏}
ここまでの例で何度もでてきたが、改めて自然変換の定義を与えることにしよう。

\begin{dfn}
圏$C, D$の間の関手$F:C \to D$と$G:C \to D$の間の自然変換とは、
$C$の各対象$x$に対して定まる$C$の射$\phi(x):F(x)\to G(x)$の集まり$\phi=(\phi(x))_x$であって、
任意の$C$の射$f:x\to y$に対し$G(f)\circ\phi(x)=\phi(y)\circ F(f)$となる、つまり次の図式が可換になるもの。
\[
\begin{CD}
F(x)@>\phi(x)>>G(x)\\
@VF(f)VV@VG(f)VV\\
F(y)@>\phi(y)>>G(y)
\end{CD}
\]
\end{dfn}

実は、自然変換を用いて、関手の圏を定義することができる。
与えられた二つの圏$C, D$から、新しく圏$C^D$を対象が関手たち、射が自然変換たちであるようにして定める。
\begin{dfn}
圏$C, D$にたいし、圏$C^D$を
\begin{itemize}
\item 対象は関手$F:D\to C$
\item 射$\phi:F\to G$は$C$の各対象$x$に対して定まる$C$の射$\phi(x):F(x)\to G(x)$の集まり$\phi=(\phi(x))_x$であって、任意の$C$の射$f:x\to y$に対し$G(f)\circ\phi(x)=\phi(y)\circ F(f)$となる、つまり次の図式が可換になるもの。
\end{itemize}
\[
\begin{CD}
F(x)@>\phi(x)>>G(x)\\
@VF(f)VV@VG(f)VV\\
F(y)@>\phi(y)>>G(y)
\end{CD}
\]
\end{dfn}

ここまでの例で言えば、$g_{[0]}$は$\Sets^{\Sets}$の対象である。
直積を定める関手は$L_X, R_X$は$\Sets^{\Sets}$の対象であり、$t:L_X \to R_X$は$\Sets^{\Sets}$の射である。
$\times, \pr_1, \pr_2$は$\Sets^{\Sets^{[1]}}$の対象であり、$tr$は$(\Sets^{[1]})^{\Sets^{[1]}}$の対象である。

有向グラフは$\Sets^\Gamma$の対象であり、有向グラフの射は$\Sets^\Gamma$の射である。

また、冪集合を取る関手$P$は$\Sets^{\Sets^{\rm op}}$の対象であり、また$h_{[1]}$も同じく$\Sets^{\Sets^{\rm op}}$の対象である。
上の例でみた自然変換$\phi, \psi$は$\Sets^{\Sets^{\rm op}}$の射である。

\section{米田の補題}
ここでは米田の補題についてその主張と証明を紹介します。

まずは自然変換と関手の圏について復習します。
詳しくは前節に書いてあるので、そちらも合わせてお読みください。
次に表現可能関手を定義し、米田の補題の主張を紹介します。
さらに、数直線上の関数から定まる関手の例を用いて、米田の補題がどのようなことを主張しているのか見てみます。
その後、米田の補題を証明します。
できるだけ式変形を丁寧に説明しましたが、逆に長く読みづらくなってしまったかもしれません。
最後に、冪集合関手について米田の補題を応用することで、ドモルガンの定理を証明します。
ここでも、米田の補題の証明に触れるので、上の証明がわかりづらい場合にも具体例と合わせてお読みいただくとよいと思います。

\subsection{自然変換}
自然変換というのは、圏$C, D$の間の二つの関手$F, G$を結びつけるものであった。

\begin{dfn}
圏$C, D$の間の関手$F:C \to D$と$G:C \to D$を考える。
これに対して$F$と$G$の間の自然変換とは、
\begin{itemize}
\item $C$の各対象$x$に対して定まる$C$の射$\phi(x):F(x)\to G(x)$の集まり$\phi=(\phi(x))_x$
\end{itemize}
であって、任意の$C$の射$f:x\to y$に対し$G(f)\circ\phi(x)=\phi(y)\circ F(f)$となる、つまり次の図式
\[
\begin{CD}
F(x)@>\phi(x)>>G(x)\\
@VF(f)VV@VG(f)VV\\
F(y)@>\phi(y)>>G(y)
\end{CD}
\]
が可換になるもの。
\end{dfn}

\begin{eg}
自然変換の例として、冪集合を与える関手$P:\Sets^{\rm op} \to \Sets$と集合$[1]=\{0,1\}$が定める関手$h_{[1]}=\Hom(-,[1]):\Sets^{\rm op} \to \Sets$の間の自然変換がある。

まず、関手$P$は
\begin{itemize}
\item 集合$X$に対しその部分集合全体$P(X)=\{A \subset X\}$を対応させる
\item 写像$f:X \to Y$から逆像により定まる写像$P(f):P(Y) \to P(X)$を対応させる
\end{itemize}
ことで定まるもの。
ここで$B \subset Y$の$f$による逆像とは$f^{-1}(B)=\{x \in X, f(x)\in B\}$のことである。
つまり$f:X \to Y$の$P$により定まる写像$P(f)$は$P(f)(B)= f^{-1}(B)$で定まるものである。

次に、関手$h_{[1]}$は
\begin{itemize}
\item 集合$X$に対して$X$から$[1]$への写像全体のなす集合$\Hom(X,[1])=\{f:X \to [1]\}$を対応させる
\item 写像$f:X \to Y$に対して$g \mapsto g \circ f$により定まる写像$\Hom(Y,[1]) \to \Hom(X,[1])$を対応させる
\end{itemize}
ことで定まるもの。

集合$h_{[1]}(X)$の特別な元として特性関数がある。
これは$X$の部分集合$A$を用いて定義される写像$X \to [1]$で、
\begin{align*}
\chi_A(x)=\begin{cases}1 & (x\in A) \\ 0 & (x\notin A)\end{cases}
\end{align*}
で定まるものである。

この特性関数を用いて、上の二つの関手$P, h_{[1]}$の間の自然変換を定めよう。

まず各集合$X$に対して写像$\phi(X):P(X) \to \Hom(X,[1])$を
\begin{align*}
\phi(X)(A) = \chi_A
\end{align*}
により定める。

このとき、集合$X, Y$と写像$f:X \to Y$に対して図式
\[
\begin{CD}
P(Y)@>\phi(y)>>h_{[1]}(Y)\\
@VP(f)VV@Vh_{[1]}(f)VV\\
P(X)@>\phi(x)>>h_{[1]}(X)
\end{CD}
\]
が可換になることが以下のようにわかる。
各$A \subset Y$に対して、$x\in f^{-1}(A)$であることと$f(x)\in A$であることが同値であることを使うと、
写像$X \to [1]$としての等式
\begin{align*}
\phi(X)P(f)(A)=\chi_{f^{-1}(A)}=\chi_A\circ f=h_{[1]}(f)(\phi(Y)(A))
\end{align*}
が成立する。
\end{eg}


次に二つの圏$C, D$について、$C$から$D$への関手全体のなすの圏について復習しよう。
これは対象が関手、射が自然変換として定まる圏である。
\begin{dfn}
圏$C, D$にたいし、圏$C^D$を
\begin{itemize}
\item 対象は関手$F:D\to C$
\item 射$\phi:F\to G$は$F$から$G$への自然変換。つまり、$C$の各対象$x$に対して定まる$C$の射$\phi(x):F(x)\to G(x)$の集まり$\phi=(\phi(x))_x$であって、任意の$C$の射$f:x\to y$に対し$G(f)\circ\phi(x)=\phi(y)\circ F(f)$となる、つまり次の図式が可換になるもの。
\end{itemize}
\[
\begin{CD}
F(x)@>\phi(x)>>G(x)\\
@VF(f)VV@VG(f)VV\\
F(y)@>\phi(y)>>G(y)
\end{CD}
\]
ここで、自然変換$\phi:F \to G$と$\psi:G\to H$の合成は、$(\psi\circ\phi)(x)=\psi(x)\circ \phi(x)$により与えられる。
以下の図式で、一番外側の長方形が可換であることは小さい正方形が可換であることから言える。
\[
\begin{CD}
F(x)@>\phi(x)>>G(x)@>\psi(x)>>H(x)\\
@VF(f)VV@VG(f)VV@VH(f)VV\\
F(y)@>\phi(y)>>G(y)@>\psi(y)>>H(y)
\end{CD}
\]
また、恒等射$\id_F$はそれぞれの$x$について$\id_F(x)=\id_{F(x)}$とすれば定まる。
\end{dfn}

上の例で与えた$P, h_{[1]}$はいずれも$\Sets^{C^{\rm op}}$の対象であり、
特性関数を用いて定めた自然変換は$\Sets^{C^{\rm op}}$の射である。

\subsection{表現可能関手}
圏$C$の対象$x$から関手$h_x:C^{\rm op} \to \Sets$を定めることができる。
この関手は
\begin{itemize}
\item 対象の対応が$y \mapsto \Hom_C(y,x)$
\item 射の対応が$C$の射$f:y \to z$に対して$h_x(f):h_x(z) \to h_x(y)$
\end{itemize}
により定まるものである。

\begin{dfn}
関手$F:C^{\rm op}\to\Sets$が表現可能とは、ある$C$の対象$x$に対して$F$と$h_x$が同型になることをいう。
ここで同型というのは関手の圏$\Sets^{C^{\rm op}}$においての同型ということであり、
すなわち自然変換$\phi:F \to h_x$と$\psi:h_x \to F$が存在して$\phi\circ\psi=\id_{h_x}$と$\psi\circ\phi=\id_F$が成り立つことである。
\end{dfn}

\begin{eg}
$P$は$[1]$により表現される関手$h_{[1]}$と同型になる、すなわち関手$P$は表現可能関手であることを確かめよう。

まず上で定義した特性関数を用いた写像$\phi(X):P(X) \to h_{[1]}(X),  \phi(X)(A) = \chi_A$により、自然変換$\phi:P \to h_{[1]}$が定まる。

これと逆向きの自然変換$\psi:h_{[1]} \to P$を定めよう。

まず集合$X$に対して$\psi(X):h_{[1]}(X) \to P(X)$を次のようにして定める。
$h_{[1]}$の元は写像$f:X \to [1]$であり、
これに対して部分集合を$1$にうつる要素全体$f^{-1}(1)=\{x \in X \mid f(x)=1\} \subset X$を対応させる。
つまり、
\begin{align*}
\psi(X)(f)=f^{-1}(1)
\end{align*}
としよう。

このとき、$\Sets$の射$f:Y \to X$に対して図式
\begin{align*}
\begin{CD}
h_{[1]}(Y)@>\psi(Y)>>P(Y)\\
@Vh_{[1]}(f)VV@VP(f)VV\\
h_{[1]}(X)@>\psi(X)>>P(X)
\end{CD}
\end{align*}
が可換になることを確かめれば、これが自然変換を与えることがわかる。

$g \in h_{[1]}(Y)$すなわち写像$g:Y \to [1]$をとる。
右上を通って$P(X)$にうつすと、まず$\psi(Y)(g)=g^{-1}(1)$であり、
\begin{align*}
P(f)(g^{-1}(1))=f^{-1}(g^{-1}(1))=\{x \in X \mid f(x) \in g^{-1}(1)\}=\{x \in X \mid g(f(x))=1\} \in P(X)
\end{align*}
となる。

一方で左下を通って$P(X)$にうつすと、
$h_{[1]}(f)(g)=g\circ f:X \to [1]$であり、
\begin{align*}
\psi(X)(g\circ f)=(g\circ f)^{-1}(1)=\{x \in X \mid (g\circ f)(x) = 1\}
\end{align*}
となることから、二つの部分集合が一致し、図式が可換であることがわかる。

この$\phi(X):P(X)\to h_{[1]}(X)=\Hom_{\Sets}(X,[1])$と$\psi(X):h_{[1]}(X)=\Hom_{\Sets}(X,[1])\to P(X)$が互いに逆であることを確かめる。

まず$\psi(X)\circ\phi(X)$を計算する。
$A \subset X$に対して、$\phi(X)(A)=\chi_A$であり、
$x\in\chi^{-1}_X(1)$であることは$\chi_A(x)=1$であることと同値で、これは$x\in A$と同値なので、
\begin{align*}
\psi(X)(\chi_A)=\chi_A^{-1}(1)=A
\end{align*}
である。
よって$\psi(X)\circ\phi(X)=\id_{P(X)}$である。

次に$\phi(X)\circ\psi(X)$を計算する。
$f:X\to [1]$に対して、$\psi(X)(f)=f^{-1}(1)$であり、
$\chi_{f^{-1}(1)}(x)=1$であることは$x\in f^{-1}(1)$であることと同値で、これは$f(x)=1$であることと同値なので、
\begin{align*}
\phi(f^{-1}(1))=\chi_{f^{-1}(1)}=f
\end{align*}
である。
よって$\phi(X)\circ\psi(X)=\id_{h_{[1]}(X)}$である。
\end{eg}

$C$の射$f:x \to y$があれば関手$h_x$と$h_y$の間の自然変換$h_f:h_x\to h_y$が次のようにして定まる。
\begin{itemize}
\item $C$の各対象$z$に対し写像$h_f(z):\Hom_C(z,x)=h_x(z)\to \Hom_C(z,y)=h_y(z)$を$g:z \to x$に対して$f\circ g:z \to y$を対応させるものとして与える
\item $C$の射$g:z\to w$について写像の等式$h_f(z)\circ h_x(g)=h_y(g)\circ h_f(w)$、つまり次の図式
\[
\begin{CD}
h_x(w)@>h_f(w)>>h_y(w)\\
@Vh_x(g)VV@Vh_y(g)VV\\
h_x(z)@>h_f(z)>>h_y(z)
\end{CD}
\]
が可換になる。
実際、$h:w\to x \in h_x(w)$に対して、左下にうつすと
\begin{align*}
h_x(g)(h)=h\circ g:z \to x
\end{align*}
となり、さらに右下にうつすと
\begin{align*}
h_f(z)(h_x(g)(h))=h_f(z)(h\circ g)=f\circ h \circ g:z \to y
\end{align*}
となる。
一方、$h$を右上にうつすと
\begin{align*}
h_f(w)(h)=f\circ h:w \to y
\end{align*}
となり、さらに右下にうつすと
\begin{align*}
h_y(g)(h_f(w)(h))=h_y(g)(f\circ h)=f\circ h \circ g:z \to y
\end{align*}
となり、この二つは一致することから上の図式は可換であることがわかる。
\end{itemize}

これを用いて米田埋め込みと呼ばれる関手$C\to \Sets^{C^{\rm op}}$を定義しよう。
\begin{dfn}[米田埋め込み]
関手$Y_C:C\to \Sets^{C^{\rm op}}$を
\begin{itemize}
\item 対象$x$に対し関手$h_x$を対応させる
\item $C$の射$f:x\to y$に自然変換$h_f:h_x\to h_y$を対応させる
\end{itemize}
ことで定義する。
これを米田埋め込みという。
\end{dfn}

\vspace{5pt}

これが埋め込みと言われる理由を説明するのが次の米田の補題である。
簡単にいうと、圏$C$を圏$\Sets^{C^{\rm op}}$の中に埋め込んで調べることができるということを意味している。

このことを理解するために、まずは$C$の対象$x$たちを埋め込んだ関手$h_x$たちと他の関手$F$の関係を調べよう。
関手$F:C^{\rm op} \to \Sets$を一つ固定する。
また関手$G:C^{\rm op} \to \Sets$を
\begin{itemize}
\item 対象$x$を集合$\Hom_{\Sets^{C^{\rm op}}}(h_x,F)$にうつす
\item $C$の射$f:x\to y$を写像$(h_f)^*:\Hom_{\Sets^{C^{\rm op}}}(h_y,F)\to \Hom_{\Sets^{C^{\rm op}}}(h_x,F)$にうつす
\end{itemize}
により定義する。
$(h_f)^*$の定義は複雑だが、これは自然変換$\phi:h_y \to F$を自然変換$h_f:h_x \to h_y$と合成して得られる自然変換$\phi\circ h_f$に対応させるものである。

このとき、この関手$G$と元の関手$F$が自然同型になるというのが米田の補題である。
つまり、$h_x$から$F$への自然変換は$F(x)$の要素と一対一に対応し、しかもこの対応は自然であるということを言っている。
\begin{thm}[米田の補題]
関手の同型$\phi:F\to G$が存在する。
すなわち、$C$の任意の対象$x$に対して全単射$\phi(x):F(x) \to G(x)=\Hom(h_x,F)$が存在し、
$C$の射$f:x \to y$に対して次の図式が可換になる。
\[
\begin{CD}
F(y)@>\phi(y)>>G(y)=\Hom_{\Sets^{C^{\rm op}}}(h_y,F)\\
@VF(f)VV@VG(f)VV\\
F(x)@>\phi(x)>>G(x)=\Hom_{\Sets^{C^{\rm op}}}(h_x,F)
\end{CD}
\]
\end{thm}

実際に証明する前に、ごく簡単な例で米田の補題について考えてみよう。

\begin{eg}
数直線$\R$の開区間$I$に対し$F(I)$を$I$が定義域であるような実数値関数全体の集合とする。
$f \in F(I)$と$J\subset I$に対し、$f$を$J$に制限することで$f\vert_J \in F(J)$が定まり、
これにより写像$F(I) \to F(J)$を定めることができる。
このようにして全ての$J \subset I$に対して一斉に$f\vert_J$を定めることができ、しかもこれらは互いに制限の関係$(f\vert_J)\vert_{J'}=f\vert_{J'}$になっている。

逆に$I$に含まれる区間$J$全てに対して$f_J$が与えられていて、しかも$J\subset J'$について制限の関係$f_J\vert_{J'}=f_{J'}$になっていれば、これはある$I$上の関数を制限したものとして書くことができる。
これはある意味では当たり前で、なぜなら$J=I$についての$f_I$が元の$f$を与える。
つまり、
\begin{align*}
\phi(I):F(I) \to \{(f_J)_{J\subset I}\vert~ f_J\vert_{J'} = f_{J'}, \forall J' \subset J\}=G(J)
\end{align*}
が全単射であるということである。
さらにこの全単射は制限$I'\subset I$について整合的である、つまり$J'\subset J$に対して図式
\[
\begin{CD}
F(J)@>\phi(J)>>G(J)=\Hom_{\Sets^{C^{\rm op}}}(h_y,F)\\
@VVV@VVV\\
F(J')@>\phi(J')>>G(J)=\Hom_{\Sets^{C^{\rm op}}}(h_x,F)
\end{CD}
\]
が可換である。

これは特別な圏と関手についての米田の補題とみなすことができる。
$C$を$\R$の開区間を対象とし、包含を射とする圏を考える。
関手$F:C^{\rm op} \to \Sets$を$F(I)$を$I$上の実数値関数、射は制限で定める。
$I$が表現する関手$h_I$は$J \subset I$に対しては一点集合、$J \not\subset I$に対しては空集合を与える関手$C^{\rm op} \to \Sets$である。
このとき、自然変換$h_I \to F$とは、各開区間$J$に対して$J \subset I$について$h_I(J)=\{*\} \to F(J)$を与え、$J'\subset J$に対して次の図式が整合的になる。
\[
\begin{CD}
h_I(J)@>\phi(y)>>F(J)\\
@VVV@VVV\\
h_I(J')@>\phi(x)>>F(J')
\end{CD}
\]
写像$h_I(J) \to F(J)$を与えるのは$f_J\in F(J)$を一つ与えることと同じで、
上の図式の可換性は$f_J\vert_{J'}=f_{J'}$であるということを主張する。
したがって、米田の補題の同型は、上で与えた
\begin{align*}
F(I) \to \{(f_J)_{J\subset I}\vert~ f_J\vert_{J'} = f_{J'}, \forall J' \subset J\}
\end{align*}
が全単射であることと同じことを言っていることになる。
\end{eg}

\subsection{証明}

二つの関手$F,G:C^{\rm op} \to \Sets$は同型、すなわち関手として自然同値であることを以下の手順に従って証明しよう。
\begin{enumerate}
\item 自然変換$\phi:F \to G$を定義する。
\item 自然変換$\psi:G \to F$を定義する。
\item $\phi$と$\psi$が互いに逆であることを確かめる。
\end{enumerate}

\begin{proof}
上の手順に沿って証明を進めていく。
\begin{enumerate}
\item 自然変換$\phi:F\to G$を定義するため、
\begin{enumerate}
\item $C$の対象$x$に対して写像$\phi(x):F(x) \to G(x)$を定義
\item $C$の射$g:x\to y$に対し図式
\begin{align*}
\begin{CD}
F(y)@>\phi(y)>>G(y)=\Hom_{\Sets^{C^{\rm op}}}(h_y,F)\\
@VF(f)VV@VG(f)VV\\
F(x)@>\phi(x)>>G(x)=\Hom_{\Sets^{C^{\rm op}}}(h_x,F)
\end{CD}
\end{align*}
が可換である
\end{enumerate}
ことを示す。
\begin{enumerate}
\item 写像$\phi(x):F(x)\to G(x)=\Hom_{\Sets^{C^{\rm op}}}(h_x,F)$を定めるため、
各$a\in F(x)$に対して$G(x)$の元すなわち自然変換$\phi(x)(a):h_x\to F$を定める。
\begin{enumerate}
\item $C$の対象$y$について写像$(\phi(x)(a))(y):h_x(y)=\Hom_C(y,x)\to F(y)$を$f:y\to x$に対して$F(f):F(x)\to F(y)$による$a$の像$F(f)(a)\in F(y)$を対応させるものとする。つまり
\begin{align}\label{phi_def}
((\phi(x)(a))(y))(f)=F(f)(a)
\end{align}
である。
\item これが$C$の射$g:y\to z$について以下の図式
\begin{align}\label{phixa}
\begin{CD}
h_x(z)@>\phi(x)(a)(z)>>F(z)\\
@Vh_x(g)VV@VF(g)VV\\
h_x(y)@>\phi(x)(a)(y)>>F(y)
\end{CD}
\end{align}
を可換にする、つまり
\begin{align}\label{phixa_comm}
(\phi(x)(a))(y)\circ h_x(g)=F(g)\circ(\phi(x)(a))(z)
\end{align}
を満たすことを確かめる。

まず$f\in h_x(z)=\Hom_C(z,x)$をとる。
左下にうつすと
\begin{align*}
h_x(g)(f) = f\circ g:y \to x
\end{align*}であり、さらに右下にうつすと(\ref{phi_def})より
\begin{align}\label{phixa_ld}
(\phi(x)(a)(y))(h_x(g)(f))=(\phi(x)(a)(y))(f\circ g)=F(f\circ g)(a)
\end{align}
となる。

一方、$f$を右上にうつすと
\begin{align*}
(\phi(x)(a)(z))(f) = F(f)(a)
\end{align*}
であり、さらに右下にうつすと
\begin{align}\label{phixa_ru}
F(g)((\phi(x)(a)(z))(f)) = F(g)(F(f)(a)) =(F(g)\circ F(f))(a)
\end{align}
となる。

ここで$F$が$C^{\rm op}$から$\Sets$の関手であるから$F(g)\circ F(f)=F(f \circ g)$が成り立つ。
これを使うと
\begin{align*}
(F(g)\circ F(f))(a)=F(f \circ g)(a)
\end{align*}
となるから(\ref{phixa_ld})と(\ref{phixa_ru})の右辺が等しく、(\ref{phixa_comm})が示せた。
つまり図式(\ref{phixa})が可換であることがわかった。
\end{enumerate}
以上から$\phi(x)(a)$は自然変換であり、
$C$の各対象$x$について$\phi(x):F(x)\to G(x)$という写像が定義できた。

\item このように構成した$\phi(x)$たちが自然変換を定めることを見よう。$C$の射$g:x\to y$に対し図式
\begin{align}\label{phi_comm}
\begin{CD}
F(y)@>\phi(y)>>G(y)=\Hom_{\Sets^{C^{\rm op}}}(h_y,F)\\
@VF(g)VV@VG(g)VV\\
F(x)@>\phi(x)>>G(x)=\Hom_{\Sets^{C^{\rm op}}}(h_x,F)
\end{CD}
\end{align}
が可換になる、つまり$\phi(x)\circ F(g)=G(g)\circ\phi(y)$が成り立つことを確かめる。

まず$a\in F(y)$をとる。
この$a$を左下を通って右下にうつすと$\phi(x)(F(g)(a))$は自然変換$h_x\to F$であり、
これは$C$の各対象$z$に対して写像$\phi(x)(F(g)(a))(z):h_x(z) \to F(z)$を(\ref{phi_def})により$f\in h_x(z)$に対して
\begin{align}\label{phix_ld}
(\phi(x)(F(g)(a))(z))(f)=F(f)(F(g)(a))
\end{align}
により定めるものである。

一方、右上にうつすと$\phi(y)(a)$は自然変換$h_y \to F$であり、
これは$C$の各対象$z$に対して写像$(\phi(y)(a))(z):h_y(z) \to F(z)$を(\ref{phi_def})により
\begin{align*}
(\phi(y)(a))(z)(f)=F(f)(a)
\end{align*}
により定めるものである。
この$\phi(y)(a)$をさらに右下にうつす。
$G(g)$は自然変換に$h_g$を合成することで定まる写像であることから、
\begin{align*}
G(g)(\phi(y)(a))=\phi(y)(a)\circ h_g
\end{align*}
である。
これにより$C$の各対象$z$に対し、写像$(\phi(y)(a))(z)\circ h_g(z)$が$h_x(z) \to  F(z)$として、
$h_y$の定義および(\ref{phi_def})を用いて
\begin{align}\label{phix_ru}
(\phi(y)(a))(z)\circ h_g(z)(f) = (\phi(y)(a))(z)(h_g(f))=(\phi(y)(a))(z)(g\circ f)=F(g\circ f)(a)
\end{align}
と計算できる。
ここで$F$が$C^{\rm op}$から$\Sets$の関手であるから$F(g)\circ F(f)=F(f \circ g)$が成り立つ。
これを使うと
\begin{align*}
(F(g)\circ F(f))(a)=F(f \circ g)(a)
\end{align*}
となり、(\ref{phix_ld})および(\ref{phix_ru})により
\begin{align*}
(\phi(x)(F(g)(a))(z))(f)=(\phi(y)(a))(z)\circ h_g(z)(f)
\end{align*}
となる。これがすべての$f\in h_x(z)$について成立するので、写像$h_x(z) \to F(z)$としての等式
\begin{align*}
(\phi(x)(F(g)(a))(z))=(\phi(y)(a))(z)\circ h_g(z)
\end{align*}
が成り立つ。
さらに、これがすべての$C$の対象$z$について成立するので、自然変換$h_x \to F$としての等式
\begin{align*}
\phi(x)(F(g)(a))=(\phi(y)(a))\circ h_g=G(g)(\phi(y)(a))\\
\end{align*}
が成り立ち、図式(\ref{phi_comm})が可換であることがわかった。
\end{enumerate}

\item 自然変換$\psi:G\to F$を定義しよう。
このために、
\begin{enumerate}
\item $C$の対象$x$に対して写像$\psi(x):G(x) \to F(x)$を定義し、
\item $C$の射$f:y\to x$に対し図式
\[
\begin{CD}
G(y)=\Hom_{\Sets^{C^{\rm op}}}(h_y,F)@>\psi(y)>>F(y)\\
@VF(f)VV@VG(f)VV\\
G(x)=\Hom_{\Sets^{C^{\rm op}}}(h_x,F)@>\psi(x)>>F(x)
\end{CD}
\]
を可換にする
\end{enumerate}
ことを示す。
\begin{enumerate}
\item 写像$\psi(x):G(x)=\Hom_{\Sets^{C^{\rm op}}}(h_x,F)\to F(x)$を定める。
$G(x)$の元$f$は$h_x$から$F$への自然変換なので、$C$の各対象$y$にたいし$f(y):h_x(y)\to F(y)$が定まっている。
特に$y=x$として$f(x):h_x(x)=\Hom_C(x,x)\to F(x)$が取れるので、この写像を用いて
$\id_x\in\Hom_C(x,x)$をうつすことで$\psi(x)$を定める。
つまり
\begin{align}\label{psi_def}
\psi(x)(f)=f(x)(\id_x) \in F(x)
\end{align}
と定める。
\item $C$の射$g:x\to y$に対し
\begin{align}\label{psi_comm}
\begin{CD}
G(y)=\Hom_{\Sets^{C^{\rm op}}}(h_y,F)@>\psi(y)>>F(y)\\
@VG(g)VV@VF(g)VV\\
G(x)=\Hom_{\Sets^{C^{\rm op}}}(h_x,F)@>\psi(x)>>F(x)
\end{CD}
\end{align}
が可換、すなわち
\begin{align*}
\psi(x)\circ G(g)=F(g)\circ\psi(y)\in F(x)
\end{align*}
を満たすことを確かめる。

まず自然変換$f\in G(y)$をとる。
これを左下にうつすと、自然変換$G(g)(f):h_x \to F$は$f:h_y \to F$と自然変換$h_g:h_x \to h_y$を合成したもの、つまり
\begin{align*}
G(g)(f)=f\circ h_g:h_x \to F
\end{align*}
である。

これを右下に$\psi(x)$でうつすと(\ref{psi_def})より
\begin{align}\label{psi_ld}
\psi(s)(G(g)(f))=(f\circ h_g)(x)(\id_x)=f(x)\circ(h_g(x)(\id_x)
\end{align}
にうつる。

一方で、$f$をまず右上にうつすと(\ref{psi_def})より
\begin{align*}
\psi(y)(f)=f(y)(\id_y)
\end{align*}
である。
これをさらに$F(g)$で右下にうつした$F(g)(f(y)(\id_y))$を計算しよう。
$f$は自然変換$h_y \to F$であるから、次の図式
\[
\begin{CD}
h_y(y)@>f(y)>>F(y)\\
@Vh_y(g)VV@VF(g)VV\\
h_y(x)@>f(x)>>F(x)
\end{CD}
\]
は可換であり、$f(x)\circ h_y(g)=F(g)\circ f(y)$が成り立つ。
したがって
\begin{align}\label{psi_ru}
F(g)(f(y)(\id_y))=f(x)(h_y(g)(\id_y))
\end{align}
となる。

ここで、$h_y$の定義より
\begin{align*}
h_y(g)(\id_y)=\id_y\circ g=g
\end{align*}
であり、$h_g$の定義より
\begin{align*}
h_g(x)(\id_x)=g\circ\id_x=g
\end{align*}
である。

よって、(\ref{psi_ld})と(\ref{psi_ru})により図式(\ref{psi_comm})の可換性が証明できた。
\end{enumerate}

\item 二つの自然変換$\phi, \psi$が互いに逆であることを確かめよう。
つまり$C$の各対象$x$に対し、$\phi(x)$と$\psi(x)$が$F(x)$と$G(x)$の間の逆写像を与えていることを確かめる。

\begin{enumerate}
\item $\phi(x)\circ\psi(x)$を計算する。
$G(x)=\Hom_{\Sets^{C^{\rm op}}}(h_x,F)$の元$f$をとる。
$f$は$h_x$から$F$への自然変換であり、(\ref{psi_def})から
\begin{align*}
\psi(x)(f)=f(x)(\id_x)\in F(x)
\end{align*}
である。
さらに$\phi(x)(\psi(x)(f))=\phi(x)(f(x)(\id_x))$は自然変換$h_x \to F$である。
この自然変換が元の$f$と一致することを確かめよう。
これは、$C$の各対象$y$について写像$h_x(y) \to F(y)$として一致すればよく、
$g\in h_x(y)=\Hom_C(y,x)$に対して(\ref{phi_def})より
\begin{align*}
(\phi(x)\psi(x)(f))(y)(g)=\phi(x)(f(x)(\id_x))(y)(g)=F(g)(f(x)(\id_x))
\end{align*}
である。
ここで$f$が自然変換であるから図式
\[
\begin{CD}
h_x(x)@>f(x)>>F(x)\\
@Vh_x(g)VV@VF(g)VV\\
h_x(y)@>f(y)>>F(y)
\end{CD}
\]
が可換である。
したがって
\begin{align*}
F(g)(f(x)(\id_x))=f(y)(h_x(g))(\id_x)=f(y)(g)
\end{align*}
となり
\begin{align*}
(\phi(x)\psi(x)(f))(y)(g)=f(y)(g)
\end{align*}
である。

これが各$g\in h_x(y)$について成り立つので$h_x(y) \to F(y)$として
\begin{align*}
\phi(x)\psi(x)(f)(y)=f(y)
\end{align*}
であり、
これが$C$の各対象$y$について成り立つので
\begin{align*}
\phi(x)\psi(x)(f)=f=\id_{G(x)}(f)=\id_G(x)(f)
\end{align*}
である。
これが各$f, x$について成り立つので、自然変換として$\phi\circ\psi=\id_G$である。

\item $\psi(x)\circ\phi(x)$を計算する。
$F(x)$の元$a$をとる。
$\phi(x)(a)$は自然変換$h_x \to F$である。
これを$\psi(x)$でうつすと、(\ref{psi_def})と(\ref{phi_def})により
\begin{align*}
\psi(x)(\phi(x)(a))=(\phi(x)(a))(x)(\id_x)=F(\id_x)(a)
\end{align*}
であり、さらに$F$が関手であることから
\begin{align*}
F(\id_x)(a)=\id_{F(x)}(a)
\end{align*}
となる。

したがって任意の$a\in F(x)$に対して
\begin{align*}
\psi(x)(\phi(x)(a))=\id_{F(x)}(a)
\end{align*}
であるので、写像$F(x)\to F(x)$として$\psi(x)\circ\phi(x)=\id_{F(x)}=\id_F(x)$であり、
自然変換として$\psi\circ\phi=F$となる。
\end{enumerate}

以上から$\phi$と$\psi$が自然変換として互いに逆であることが証明できた。
\end{enumerate}

これにより$F$と$G$が$\Sets^{C^{\rm op}}$において同型であることが証明できた。
\end{proof}

このことから特に$F=h_y$とすると、
\begin{thm}
$Y_C:h_y(x)=\Hom_C(x,y)\to \Hom_{\Sets^{C^{\rm op}}}(h_x,h_y)$は全単射。
\end{thm}
であることがわかる。

例えば圏$C$の対象$x,y$の間に射を定義したいとき、
関手$h_x, h_y$の間に射を定義すればよいということが上の定理から言える。
一般的には圏$C$の射を定義するよりは関手の間に射を定義する方がやさしい。
なぜなら、$h_x, h_y$は集合に値を持つ関手なので、射の定義を集合の言葉で書くことができるからである。
また、それらの射が一致することも、$h_x, h_y$の射として一致することを確かめればよいことも主張している。
このようにして圏$C$の対象の性質を集合の言葉で理解することができる。

逆に$C$の対象$x$が定める関手$h_x$の性質を$x$を調べることで理解することができる。
例えば$F$が何らかの興味ある関すだとして、これが$x$により表現可能、つまり$F=h_x$であるとしよう。
このとき、$F$の性質を$x$の性質を用いて調べることができる。
このような例として次の節でドモルガンの定理の証明を紹介する。

\subsection{応用}
$h_{[1]}$は冪集合を取る関手$P$と同型であることはすでに見た。

集合$X$に対して、その部分集合の補集合をとる操作は写像$c(X):P(X)\to P(X)$を定める。
これは実は関手$P$から$P$への自然変換$c:P \to P$を与えることがわかる。
つまり、$\Sets$の射$f:X \to Y$に対して以下の図式
\begin{align*}
\begin{CD}
P(Y)@>c(Y)>>P(Y)\\
@VP(f)VV@VP(f)VV\\
P(X)@>c(X)>>P(X)
\end{CD}
\end{align*}
が可換であることが以下のようにわかる。

$A \in P(Y)$すなわち部分集合$A\subset Y$をとる。
右上を通って$P(X)$にうつすと、まず$c(Y)(A)=Y \setminus A$であり、
これに対して$P(f)(Y\setminus A)=\{x \in X \mid f(x) \in Y \setminus A\}$である。

一方で左下を通って$P(X)$にうつすと、まず$P(f)(A)=\{x \in X \mid f(x) \in A\}$であり、
$c(X)(P(f)(A))=X \setminus P(f)(A)=\{x \in X \mid x \notin P(f)(A)\}=\{x \in X \mid f(x) \notin A\} $である。
このことから、二つの部分集合が一致し、図式が可換であることがわかる。

\vspace{10pt}
先に述べたように、$P$と$h_{[1]}$は同一視できる。
この同一視により、補集合を取るという自然変換$c:P \to P$は自然変換$c:h_{[1]} \to h_{[1]}$と見ることができる。
つまり$c\in\Hom_{\Sets^{\Sets^{\rm op}}}(h_{[1]},h_{[1]})$である。

米田の補題から$\Hom_{\Sets^{\Sets^{\rm op}}}(h_{[1]},h_{[1]})=\Hom_{\Sets}([1],[1])$なので、$c$に対応する写像$[1]\to [1]$がある。
これはどのような写像になるか考えよう。
米田の補題の証明にある自然変換$\psi:G \to F$を計算する。
いま、$C=\Sets, F=h_{[1]}$として、自然変換の$[1]$での写像$\psi([1]):G([1]) \to F([1])$を調べると、
\begin{align*}
\psi([1]):G([1])=\Hom_{\Sets^{\Sets^{\rm op}}}(h_{[1]},h_{[1]}) \to F([1])=\Hom_{\Sets}([1],[1])
\end{align*}
は(\ref{psi_def})により$f$を補集合をとる自然変換$c$として
\begin{align*}
\psi([1])(c)=c([1])(\id_{[1]})\in\Hom_{\Sets}([1],[1])
\end{align*}
である。

図式
\begin{align*}
\begin{CD}
h_{[1]}([1])@>\psi([1])>>P([1])\\
@VcVV@VcVV\\
h_{[1]}([1])@>\psi([1])>>P([1])
\end{CD}
\end{align*}
を用いて$P([1])$で$c([1])(\id_{[1]})$に対応する集合を計算すると、
$\id_{[1]}$に対応するのが$\id_{[1]}^{-1}(1)=\{1\}\in P([1])$であり、
これの補集合だから$c([1])(\id_{[1]})$に対応する集合は$\{0\}$である。
これをもう一度$h_{[1]}([1])$に戻すと、$\chi_{\{0\}}$すなわち$0\mapsto 1, 1\mapsto 0$で定まる写像である。

つまり、$0$と$1$を入れ替える写像$c:[1] \to [1]$から定まる自然変換$h_c:h_{[1]} \to h_{[1]}$こそが、上の自然変換$c:h_{[1]} \to h_{[1]}$であることがわかる。

\vspace{10pt}


集合$[1]$の直積$[1]\times [1]$が定める関手$h_{[1]\times[1]}:X\mapsto\Hom(X,[1] \times [1])$は
$X\mapsto P(X)\times P(X)=\{(A,B)\mid A\subset X, B\subset X\}$を与える、つまり$X$に対して$X$の部分集合の組全体のなす集合を対応させる関手と自然に同一視できることを見よう。

写像$f:X \to [1] \times [1]$というのは$f(x)=(0,0), (0,1), (1,0), (1,1)$のいずれかであり、
各成分ごとに見れば$X \to [1]$を二つ$f_1, f_2$と並べたものである。
したがって$h_{[1]}$と$P$の同一視と同じようにそれぞれの写像に対して部分集合$f_1^{-1}(1)$と$f_2^{-1}(1)$をとってやることで$X$の部分集合の組が得られる。
逆に、$X$の部分集合の組$(A, B)$があれば特性関数を並べて$(\chi_A, \chi_B):X \to [1] \times [1]$を定めることができる。
これらによって互いに逆な自然変換が与えられることは、以前と同様に証明できるのでここでは省略する。
\vspace{10pt}

さて、共通部分$A\cap B$を取る操作は写像$P(X) \times P(X) \to P(X)$を定めるが、これは$P\times P$から$P$への自然変換$\cap$を定める。
和集合$A\cup B$についても同様に自然変換$\cup:P \times P \to P$を定める。

上で見たように、これらは自然変換$h_{[1]\times[1]} \to h_{[1]}$とみなすことができ、
米田の補題の全射性から、何らかの写像$[1]\times [1] \to [1]$により与えられることがわかる。
これは具体的にはどのような写像か、補集合の場合と同様に考えてよう。

まずは$\cap$の場合を考える。
これも上と同様に、図式
\begin{align*}
\begin{CD}
h_{[1]}([1]\times[1])@>\psi([1]\times[1])>>P([1]\times[1])\times P([1]\times[1])\\
@V\cap([1]\times[1])VV@VP(f)VV\\
h_{[1]}([1]\times[1])@>\psi([1]\times[1])>>P([1]\times[1])
\end{CD}
\end{align*}
を用いて$\cap:P\times P \to P$に対応する自然変換$\cap:h_{[1]\times[1]} \to h_{[1]}$の$x=[1]\times[1]$における写像\begin{align*}
\cap([1]\times[1]):h_{[1]\times[1]}([1]\times[1]) \to h_{[1]}([1]\times[1])
\end{align*}
により、$\id_{[1]\times[1]}$がどううつるかをみる。

これを$P$の方にうつして考えると、
\begin{align*}
\cap([1]\times[1]):P([1]\times[1])\times P([1]\times[1])\to P([1]\times[1])
\end{align*}
を考えることになる。
ここで$\id_{[1]\times[1]}\in h_{[1]\times[1]}([1]\times[1])$に対応する$[1] \times [1]$の部分集合のは、
\begin{align*}
(\{(1,0),(1,1)\},\{(0,1),(1,1)\})\in P([1]\times[1])\times P([1]\times[1])
\end{align*}
である。
これを$\cap([1]\times[1])$でうつすと$\{(1,0),(1,1)\} \cap \{(0,1),(1,1)\} = \{(1,1)\} \in P([1]\times[1])$になる。
これが$\id_{[1]\times[1]}$の$h_{[1]}([1] \times [1])$へのうつり先を$P([1]\times[1])$で見たもので、
$h_{[1]}([1]\times[1])$に戻って、対応する写像は$\chi_{\{(1,1)\}}$、つまり$(1,1)$を$1$にうつし$(1,0), (0,0), (0,1)$を$0$にうつす写像$[1]\times[1] \to [1]$である。

つまりこの写像が定める自然変換$h_{[1]\times[1]} \to h_{[1]}$が$\cap$である。

次に$\cup$の場合を考える。
これも上と同様に、図式
\begin{align*}
\begin{CD}
h_{[1]}([1]\times[1])@>\psi([1]\times[1])>>P([1]\times[1])\times P([1]\times[1])\\
@V\cup([1]\times[1])VV@VP(f)VV\\
h_{[1]}([1]\times[1])@>\psi([1]\times[1])>>P([1]\times[1])
\end{CD}
\end{align*}
を用いて$\cup:P\times P \to P$に対応する自然変換$\cup:h_{[1]\times[1]} \to h_{[1]}$の$x=[1]\times[1]$における写像\begin{align*}
\cup([1]\times[1]):h_{[1]\times[1]}([1]\times[1]) \to h_{[1]}([1]\times[1])
\end{align*}
により、$\id_{[1]\times[1]}$がどううつるかをみる。

これを$P$の方にうつして考えると、
\begin{align*}
\cup([1]\times[1]):P([1]\times[1])\times P([1]\times[1])\to P([1]\times[1])
\end{align*}
を考えることになる。
ここで$\id_{[1]\times[1]}\in h_{[1]\times[1]}([1]\times[1])$に対応する$[1] \times [1]$の部分集合のは、
\begin{align*}
(\{(1,0),(1,1)\},\{(0,1),(1,1)\})\in P([1]\times[1])\times P([1]\times[1])
\end{align*}
である。
これを$\cup([1]\times[1])$でうつすと$\{(1,0),(1,1)\} \cup \{(0,1),(1,1)\} = \{(1,0), (1,1), (0,1)\} \in P([1]\times[1])$になる。
これが$\id_{[1]\times[1]}$の$h_{[1]}([1] \times [1])$へのうつり先を$P([1]\times[1])$で見たもので、
$h_{[1]}([1]\times[1])$に戻って、対応する写像は$\chi_{\{(1,0), (1,1), (0,1)\}}$、つまり$(1,0), (1,1), (0,1)$を$1$にうつし$(0,0)$を$0$にうつす写像$[1]\times[1] \to [1]$である。

つまりこの写像が定める自然変換$h_{[1]\times[1]} \to h_{[1]}$が$\cup$である。

\vspace{10pt}
さて、ドモルガンの定理とは、次のようなものであった。
\begin{thm}[ドモルガンの定理]
集合$X$の部分集合$A, B$に対して以下が成り立つ。
\begin{align*}
\overline{A\cap B}=\overline{A}\cup\overline{B}\\
\overline{A\cup B}=\overline{A}\cap\overline{B}
\end{align*}
\end{thm}
これを上で定めた写像$c, \cap, \cup$を使って書くと、
\begin{align*}
(c\times c)(\cap(A,B))=\cup(c(A),c(B)\\
(c\times c)(\cup(A,B))=\cap(c(A),c(B)
\end{align*}
であり、これは次のような図式の可換性を言っていると解釈できる。
\begin{multicols}{2}
\[
\begin{tikzcd}
P(X) \times P(X) \ar[r, "\cap"] \ar[d, "{(c, c)}"] & P(X) \ar[d, "c"]\\
P(X) \times P(X) \ar[r, "\cup"] & P(X)
\end{tikzcd}
\]

\[
\begin{tikzcd}
P(X) \times P(X) \ar[r, "\cup"] \ar[d, "{(c, c)}"] & P(X) \ar[d, "c"]\\
P(X) \times P(X) \ar[r, "\cap"] & P(X)
\end{tikzcd}
\]
\end{multicols}
これらは、$[1], [1] \times [1]$により表現される関手の間の図式である。
つまり、
\begin{multicols}{2}
\[
\begin{tikzcd}
h_{[1]\times[1]} \ar[r, "\cap"] \ar[d, "{(c, c)}"] & h_{[1]} \ar[d, "c"]\\
h_{[1]\times[1]} \ar[r, "\cup"] & h_{[1]}
\end{tikzcd}
\]

\[
\begin{tikzcd}
h_{[1]\times[1]} \ar[r, "\cup"] \ar[d, "{(c, c)}"] & h_{[1]} \ar[d, "c"]\\
h_{[1]\times[1]} \ar[r, "\cap"] & h_{[1]}
\end{tikzcd}
\]
\end{multicols}

これらの射は上で見たように$[1] \times [1] \to [1]$から定まるものであり、
これらが一致することを見れば、米田の補題の単射性から二つの自然変換が一致することがわかる。


よって以下の図式
\begin{multicols}{2}
\[
\begin{tikzcd}
{[1] \times [1]} \ar[r, "\cap"] \ar[d, "{(c, c)}"] & {[1]} \ar[d, "c"]\\
{[1] \times [1]} \ar[r, "\cup"] & {[1]}
\end{tikzcd}
\]

\[
\begin{tikzcd}
{[1] \times [1]} \ar[r, "\cup"] \ar[d, "{(c, c)}"] & {[1]} \ar[d, "c"]\\
{[1] \times [1]} \ar[r, "\cap"] & {[1]}
\end{tikzcd}
\]
\end{multicols}
が可換であればよいが、これは直接$(0,0), (1,0), (0,1), (1,1)$の行き先を計算してやれば良い。

つまり、直接集合$A, B$を触ることなく、米田の補題と表現可能関手による解釈により、ドモルガンの定理を証明することができた。
\newpage



\section{普遍性と随伴}
極限、普遍性、随伴について。
Curry-Howard対応。
全称と存在の随伴について。トポス理論の入り口まで。

\subsection{圏同値}
関手$F:C\to D$と$G:D\to C$が与えられたとき、これらが圏同値を与えるとは

\begin{prob}
対象が$1$点で恒等射のみをもつ圏$C$
\begin{xy}
(0,0)*{\bullet}="A"
\ar @(lu,ru) "A":"A"
\end{xy}
と、対象が$2$点で次のような$4$つの射をもつ圏$D$
\begin{xy}
(0,0)*{\bullet}="A", (10,0)*{\bullet}="B"
\ar @(lu,ru) "A";"A"
\ar @<1mm> "A";"B"
\ar @<1mm> "B";"A"
\ar @(lu,ru) "B";"B"
\end{xy}
を考える。
\begin{enumerate}
\item $D$の射の合成はどのように決まるか記述せよ
\item 関手$F:C\to D$と関手$G:D\to C$であって、圏同値を与えるものを構成せよ。
\end{enumerate}
\end{prob}


\section{極限}
\begin{dfn}
$X$を順序集合とする。
\begin{itemize}
\item $x\in X$が最大元とは、任意の$y\in X$にたいし$x\geq y$となること
\item $x\in X$が最小元とは、任意の$y\in X$にたいし$x\leq y$となること
\end{itemize}

$I, X$を順序集合とし、$F:I\to X$を順序を保つ写像とする。
\begin{itemize}
\item 順序集合$\Cone(F)$を集合$\{x\in X\mid \forall i \in I, x\geq F(i)\}\subset X$に$X$と同じ順序を定めたものとして定義する。
\item $\lim F$を$\Cone(F)$の最小元と定義する。
\end{itemize}
\end{dfn}

$I=[1]$で$F:[1]\to X$を$F(0)=x, F(1)=y$で定めると、$\Cone(F)=\{z\in X\mid z\geq x, z\geq y\}$となる。
\begin{prob}
$X=\mathbb{N}_+$上に割り算で順序を定めた順序集合に対し、
\begin{itemize}
\item $\Cone(F)$は$F(0)$と$F(1)$の公約数の集合
\item $\lim F$は最大公約数
\end{itemize}
であることを確かめよ。
\end{prob}


\subsection{$C^\to$}
圏$C$の射たちを対象とみなした新しい圏$C^\to$を
\begin{itemize}
\item 対象$\Ob(C^\to)$を$C$の射全体の集合
\item 射$\phi\colon f\to g$を$C$の射の組$\phi=(\phi_0,\phi_1)$で$\phi_1\circ f=g\circ\phi_0$となるもの。
\end{itemize}
により定義する。

\begin{eg}
$C=[1]$のとき$C^\to$がどのような圏になるか調べよう。
まず$C$の射は順序を保つ射$[1]\to[1]$なので三つある。
初めの記号で言えば$\Ob(C^\to)=\{f_0, f_1, f_3\}$である。

これらの間の射は何か?
$f_0\to f_0$はどのようなものかというと、$C$の射
\end{eg}

次のような圏を考える。
\begin{itemize}
\item 対象を関手$F:[1]\to C$たち
\item 射$\phi:F\to G$を組$(\phi(0):F(0)\to G(0), \phi(1):F(1)\to G(1))$であって、$G(\to)\circ\phi(0)=\phi(1)\circ F(\to)$を満たすもの
\end{itemize}
とすると、この圏と$C^\to$は同一視できる。
\[
\begin{CD}
F(0)@>\phi(0)>>G(0)\\
@VF(\to)VV@VG(\to)VV\\
F(1)@>\phi(1)>>G(1)
\end{CD}
\]
\subsection{$C/x, x/C$}
圏$C$とその対象$x$を一つ固定する。
この時圏$C/x$を、
\begin{itemize}
\item 対象は圏$C$における射$f:y\to x$
\item 射の集合$\Hom_{C/x}(f,g)$は$C$における射$\phi:y\to z$であって$C$の射として$f=g\circ\phi$を満たすもの全体
\end{itemize}
\begin{tikzcd}
y \ar[dr] \ar[rr] & & z \ar[dl]\\
& x &
\end{tikzcd}

とし、
圏$C/x$を、
\begin{itemize}
\item 対象は圏$C$における射$f:x\to y$
\item 射の集合$\Hom_{x/C}(f,g)$は$C$における射$\phi:y\to z$であって$C$の射として$g=\phi\circ f$を満たすもの全体
\end{itemize}
とする。
\begin{tikzcd}
& x \ar[dl] \ar[dr] &\\
y \ar[rr] & & z
\end{tikzcd}

\begin{prob}
$\Delta/[1], [1]/\Delta$はどのような圏であるか説明せよ
\end{prob}

圏$C/x$は次のように解釈できる。
\begin{itemize}
\item 対象は関手$F:[1]\to C$であって、$F(1)=x$であるもの
\item 射$\phi:F\to G$を$C$の射$\phi:F(0)\to G(0)$であって、$G(\to)\circ\phi=F(\to)$を満たすもの
\end{itemize}
要するに$C$における$x$への矢印を取るということと、$[1]$からの関手$F:[1]\to C$を決めることを同一視している。
\[
\begin{CD}
F(0)@>\phi>>G(0)\\
@VF(\to)VV@VG(\to)VV\\
F(1)@=G(1)
\end{CD}
\]
\begin{prob}
圏$x/C$を上と同じように適切な圏からの関手たちのなす圏として解釈せよ
\end{prob}

\subsection{$\Cone(F)$}
圏$I,C$と関手$F:I\to C$を固定する。
これに対し圏$\Cone(F)$を次のように定義する。
\begin{itemize}
\item 対象は$C$の対象$x$と$I$の全ての対象$i,j,\ldots$に対する$C$の射$f_i:x\to F(i)$たちのリスト$(x,f_i,f_j,\ldots)$であって、すべての$I$の射$g:i\to j$に対し$F(g)\circ f_i=f_j$を満たすもの全体。
\item 射$(x,f_i,\ldots)\to(y,g_i,\ldots)$は$C$における射$\phi:x\to y$であって、すべての$i$について$g_i\circ\phi=f_i$を満たすもの。
\end{itemize}


前に見たように圏$C$の射は$[1]$からの関手とみなすことができた。
\begin{itemize}
\item 圏$[2]$から$C$ヘの関手
\item 圏$P([1])$から$C$の関手
\end{itemize}
はそれぞれ圏$C$の言葉で言うとどのようなものを表しているか説明せよ


\begin{eg}
$I=[0]=\{0\}$とし、$F\colon I\to C$を$F(0)=x, F(id_0)=id_x$で定まる関手とする。
これにたいし$\Cone(F)$がどのような圏になるか調べよう。

まず対象は$C$の対象$y$と$C$射$f\colon y\to F(0)$の組のこと。
\end{eg}

\begin{eg}
$I=[1]$とし関手$F:[1]\to C$を$F(0)=x, F(1)=y$により定まるものとしよう。
これに対し$\Cone(F)$とは
\begin{itemize}
\item $C$における対象$z$と射$f,g$の組$(z, f:z\to x, g:z\to y)$を対象
\item 射$\phi:(f:z\to x,g:z\to y)\to (f':z'\to x,g':z'\to y)$は$C$の射$\phi:z\to z'$であって$f=f'\circ\phi, g=g'\circ\phi$を満たすものである。
\end{itemize}
\begin{tikzcd}
& z \ar[d] \ar[ddr] \ar[ddl, bend right=20]&\\
& z' \ar[dr] \ar[dl]&\\
x & & y
\end{tikzcd}
\end{eg}

\subsection{始対象と終対象}
\begin{dfn}
圏$C$の対象$x$をとる。
\begin{itemize}
\item $x$が始対象であるとは、任意の$C$の対象$y$について$\Hom_C(x,y)$が一元集合であること。
\item $x$が終対象であるとは、任意の$C$の対象$y$について$\Hom_C(y,x)$が一元集合であること。
\end{itemize}
\end{dfn}
つまり、圏$C$の点$x$が始対象であるとはすべての点に向かって必ず一本だけ矢印があること、終対象であるとはすべての点から必ず一本だけ矢印があること。

順序集合$X$から作った圏$X$における始対象と終対象は$X$の最大元と最小元に相当する。

\begin{prob}
圏$\Sets, \Vect/\R$における始対象と終対象は何か?
\end{prob}

すべての圏$C$について、必ず始対象や終対象があるわけではないし、あるとしても一つではない。
しかし、始対象は存在するとすればすべて同型で、しかもその同型射はただ一つである!
\footnote{このような性質を普遍性という。}
このことは終対象についても成り立つ。

\begin{prob}
圏$C$において$x,y$がいずれも始対象である時$x$と$y$が同型であることを証明せよ。
終対象についても同様。
\end{prob}

\subsection{極限}
\begin{dfn}
圏$C, I$と関手$F:I\to C$に対し、極限$\lim F$を圏$\Cone(F)$の終対象と定義する。
\end{dfn}

例えば$I$が対象も射も空集合である圏ならば$\Cone(F)=C$で$\lim F$は$C$の終対象である。
なので極限は終対象という概念を一般化したといえる。
極限$\lim F$は$\Cone(F)$の終対象であるが、射を忘れて$C$の対象と思うこともよくある。

\begin{prob}
$F:\mathbb{N}_{>0}\to P(\R)$を$n\in \mathbb{N}_{>0}$にたいして$F(n)=[0,\dfrac{1}{n}]\in P(\R)$を対応させる関手とする。
この時$\lim F$は何か?
\end{prob}
\subsection{直積}
$I=[1]$の場合の$\Cone(F)$は以前に見たとおり。
この時$\lim F$とは、$C$の対象$p$と射の組$p_1:p\to x, p_2:p\to y$であって、
任意の$\Cone(F)$の対象$(z, f:z\to x, g:z\to y)$にたいして$p_1\circ\phi=f, p_2\circ\phi=g$となる射$\phi:z\to p$がただ一つ存在するようなものである。
\footnote{このような条件のことを普遍性という}
この$\lim F$のことを$x$と$y$の直積という。
\begin{tikzcd}
& z \ar[d] \ar[ddr] \ar[ddl, bend right=20]&\\
& z' \ar[dr] \ar[dl]&\\
x & & y
\end{tikzcd}

\begin{prob}
$C=\Sets$のとき$X\times Y=\{(x,y),x\in X,y\in Y\}$が$\lim F$であることを確かめよう
\end{prob}

\subsection{equalizer}
関手$F:\Gamma\to C$は$C$の対象$F(0)=x, F(1)=y$と$C$の射$F(s)=a,F(t)=b:x\to y$を選ぶことで定まる。
この時$\Cone(F)$は
\begin{itemize}
\item 対象は$C$の対象$z$と射$f:z\to x,g:z\to y$からなる三つ組$(z, f:z\to x, g:z\to y)$で$g=F(s)\circ f=F(t)\circ f$を満たすもの
\item 射$(z,f,g)\to (z',f',g')$は$C$の射$\phi:z\to z'$であって、$f'\circ\phi=f, g'\circ\phi=g$を満たすもの。
\end{itemize}
この$\lim F$のことを$a$と$b$のequalizerという。
\begin{center}
%\includegraphics{fig9.pdf}
\end{center}
\begin{prob}
$C=\Vect/\R$の時、$\lim F=\ker(F(s)-F(t))$であることを確かめよ
\end{prob}

\subsection{極限と関手}
関手$F:I\to C$と関手$G:C\to D$があるとき、その合成$G\circ F:I\to D$が定義できる。
このとき$G\circ F$の極限$\lim(G\circ F)$と$F$の極限の$G$での行き先$G(\lim F)$が一致するか?

\subsection{余極限}
終対象に対して始対象が存在したのと同様に、余極限という概念を定義できる。
$C/x$にたいする$x/C$と同じようにまず$\mathrm{Cocone}(F)$という圏を定義し、その始対象を余極限$\mathrm{colim}F$と定義する。
\subsection{前層}
関手$C^{\rm op}\to \Sets$のことを$C$上の前層という。
つまり$\Sets^{C^{\rm op}}$は$C$上の前層の圏である。
例えば$C=\Gamma$上の前層$X:\Gamma^{\rm op}\to \Sets$はグラフとみなせる。
つまり$X(1)$が辺の集合、$X(0)$が頂点の集合で二つの射$X(s), X(t):X(1)\to X(0)$が始点と終点を表す。

集合$X$に対し$C$上の定数前層$\underline{X}$とは
\begin{itemize}
\item $C$の対象$x$にたいし集合$X$を対応させる
\item $C$の射$f$を集合の写像$id_X:X\to X$に対応させる
\end{itemize}
により定まる前層のことをいう。

\begin{prob}
集合$(0)$から定まる定数前層$\underline{(0)}$が表現可能であるような圏$C$はどのような性質を持つか。
\end{prob}

\begin{prob}
$C=(0)$上の前層の圏は$\Sets$と同一視できることを確かめよ。
\end{prob}


圏$\Delta$を
\begin{itemize}
\item $\Ob(\Delta)=\{[0], [1], [2], \ldots\}$
\item 射の集合$\Hom_\Delta([n],[m])$は$[n]$と$[m]$の間の順序を保つ写像全てを集めたもの
\end{itemize}
で定義する。

$\Delta$上の前層、つまり関手$\Delta^{\rm op}\to \Sets$のことを単体的集合という。
例えば単体的集合$S^1$を
\begin{itemize}
\item $\Delta$の対象$[n]$に対して集合$S^1([n])$として$\Hom_\Delta([n],[1])$の定数写像二つを同一視して作る集合を対応させる
\item $\Delta$の射$f:[n]\to [m]$から射$S^1(f):S^1([m])\to S^1([n])$として$f^*\Hom_\Delta([m],[1])\to\Hom_\Delta([n],[1])$が定める射を対応させる
\end{itemize}
により定めることができる。

\section{トポス}
直観論理をトポスにより理解する。
集合のトポスが古典論理に対応する。

強制法との関係について。
強制法により連続体仮説の成り立たないZFC公理系を構成する。
二重否定位相と呼ばれる位相を用いてCohen toposと言われるトポスを作れば、
これは古典論理的でありZFC公理系をみたすモデルを作ることができて、
しかもこれは連続体仮説を満たさない。

詳しくはSGL。
p277のThe Cohen Toposから。
$\N$を自然数の集合とする。
$S$を集合論のモデルとする。
$S$における集合$B$を濃度が$P\N$より大きくなるようにとる。
次に集合論のモデル$S'$を単射$g\colon B\to P\N$を持つように作る。
$P\N=2^\N$であることから、$g$を用いて$f\colon B\times\N\to 2$を作ることができる。
$f(b,n)=(n \in g(b))$とすれば良い。
$f$を多価関数と見て、そのグラフとして$\Gamma\subset B\times\N$を
$(b,n)\in\Gamma$であることが$f(b,n)=1$と同値であるように定める。
$g$が単射であるためには$b\neq b'\in B$ならばある$n\in\N$について$f(b,n)\neq f(b',n)$であればよい。

モデル$S$にはそのような$f$は存在しないが、有限近似は存在する。
有限近似とは、有限集合$F_p\subset B\times \N$と$p\colon F_p\to 2$の組からなる。
この$(F_p,p)$を条件$p$と呼ぶ。
言い換えると、条件とは$B\times\N$の交わりを持たない二つの有限リスト$(b_i,n_i), (c_j,m_j)$であって、$p(b_i,n_i)=0, p(c_j,m_j)=1$なるもの。
これら条件はposet $P$をなす。
ここで$q\leq p$であることを$F_q\supset F_p$かつ$q\vert_{F_p}=p$であることで定めることで半順序が入る。
つまり$q\leq p$であれば$q$は$p$よりよく$f$を近似していると言える。
Cohen toposとは$Sh(P, \lnot\lnot)$のこと。
これはBoolean toposであり、選択公理をみたす。
示すべき定理は、このCohen toposにおいてある対象$K$が存在し単射の列$\N\to K \to \Omega^\N$が存在し、全射$\N\to K, K\to \Omega^\N$は存在しない。
ここで$\N$は$Sh(P,\lnot\lnot)$における自然数対象、$\Omega$は$Sh(P, \lnot\lnot)$のsubobject classifierである。


\end{document}
